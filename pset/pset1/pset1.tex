\documentclass[10pt,fleqn]{article}
\newcommand{\name}[1]{\def\psettitlename{#1}}
\newcommand{\course}[1]{\def\psettitlecourse{#1}}
\newcommand{\rsection}[1]{\def\psettitlersection{#1}}
\newcommand{\psetnum}[1]{\def\psettitlepsetnum{#1}}
%\usepackage[journal=rsc]{chemstyle}
%\usepackage{mhchem}
\usepackage{amsmath}
\usepackage{amssymb}
\usepackage{amsfonts}
\usepackage{esint}
\usepackage{bbm}
\usepackage{amscd}
\usepackage{picinpar}
\usepackage[pdftex]{graphicx}
\usepackage{tikz}
\usepackage{indentfirst}
\usepackage{wrapfig}
\usepackage{units}
\usepackage{textcomp}
\usepackage[utf8x]{inputenc}
% \usepackage{feyn}
\usepackage{feynmp}
\usetikzlibrary{
  arrows,
  calc,
  decorations.pathmorphing,
  decorations.pathreplacing,
  decorations.markings,
  fadings,
  positioning,
  shapes
}

\DeclareGraphicsRule{*}{mps}{*}{}
\newcommand{\ud}{\mathrm{d}}
\newcommand{\ue}{\mathrm{e}}
\newcommand{\ui}{\mathrm{i}}
\newcommand{\res}{\mathrm{Res}}
\newcommand{\Tr}{\mathrm{Tr}}
\newcommand{\dsum}{\displaystyle\sum}
\newcommand{\dprod}{\displaystyle\prod}
\newcommand{\dlim}{\displaystyle\lim}
\newcommand{\dint}{\displaystyle\int}
\newcommand{\fsno}[1]{{\!\not\!{#1}}}
\newcommand{\eqar}[1]
{
  \begin{align*}
    #1
  \end{align*}
}
\newcommand{\texp}[2]{\ensuremath{{#1}\times10^{#2}}}
\newcommand{\dexp}[2]{\ensuremath{{#1}\cdot10^{#2}}}
\newcommand{\eval}[2]{{\left.{#1}\right|_{#2}}}
\newcommand{\paren}[1]{{\left({#1}\right)}}
\newcommand{\lparen}[1]{{\left({#1}\right.}}
\newcommand{\rparen}[1]{{\left.{#1}\right)}}
\newcommand{\abs}[1]{{\left|{#1}\right|}}
\newcommand{\sqr}[1]{{\left[{#1}\right]}}
\newcommand{\crly}[1]{{\left\{{#1}\right\}}}
\newcommand{\angl}[1]{{\left\langle{#1}\right\rangle}}
\newcommand{\tpdiff}[4][{}]{{\paren{\frac{\partial^{#1} {#2}}{\partial {#3}{}^{#1}}}_{#4}}}
\newcommand{\tpsdiff}[4][{}]{{\paren{\frac{\partial^{#1}}{\partial {#3}{}^{#1}}{#2}}_{#4}}}
\newcommand{\pdiff}[3][{}]{{\frac{\partial^{#1} {#2}}{\partial {#3}{}^{#1}}}}
\newcommand{\diff}[3][{}]{{\frac{\ud^{#1} {#2}}{\ud {#3}{}^{#1}}}}
\newcommand{\psdiff}[3][{}]{{\frac{\partial^{#1}}{\partial {#3}{}^{#1}} {#2}}}
\newcommand{\sdiff}[3][{}]{{\frac{\ud^{#1}}{\ud {#3}{}^{#1}} {#2}}}
\newcommand{\tpddiff}[4][{}]{{\left(\dfrac{\partial^{#1} {#2}}{\partial {#3}{}^{#1}}\right)_{#4}}}
\newcommand{\tpsddiff}[4][{}]{{\paren{\dfrac{\partial^{#1}}{\partial {#3}{}^{#1}}{#2}}_{#4}}}
\newcommand{\pddiff}[3][{}]{{\dfrac{\partial^{#1} {#2}}{\partial {#3}{}^{#1}}}}
\newcommand{\ddiff}[3][{}]{{\dfrac{\ud^{#1} {#2}}{\ud {#3}{}^{#1}}}}
\newcommand{\psddiff}[3][{}]{{\frac{\partial^{#1}}{\partial{}^{#1} {#3}} {#2}}}
\newcommand{\sddiff}[3][{}]{{\frac{\ud^{#1}}{\ud {#3}{}^{#1}} {#2}}}
\usepackage{fancyhdr}
\usepackage{multirow}
\usepackage{fontenc}
%\usepackage{tipa}
\usepackage{ulem}
\usepackage{color}
\usepackage{cancel}
\newcommand{\hcancel}[2][black]{\setbox0=\hbox{#2}%
\rlap{\raisebox{.45\ht0}{\textcolor{#1}{\rule{\wd0}{1pt}}}}#2}
\pagestyle{fancy}
\setlength{\headheight}{67pt}
\fancyhead{}
\fancyfoot{}
\fancyfoot[C]{\thepage}
\fancyhead[R]
{
\psettitlename \\
\psettitlecourse{} Problem Set \psettitlepsetnum \\
\ifx\psettitlersection\empty
\else
Recitation Section \psettitlersection
\fi
}
\renewcommand{\footruleskip}{0pt}
\renewcommand{\headrulewidth}{0.4pt}
\renewcommand{\footrulewidth}{0pt}
\addtolength{\hoffset}{-1.3cm}
\addtolength{\voffset}{-2cm}
\addtolength{\textwidth}{3cm}
\addtolength{\textheight}{2.5cm}
\renewcommand{\footskip}{10pt}
\setlength{\headwidth}{\textwidth}
\setlength{\headsep}{20pt}
\setlength{\marginparwidth}{0pt}
\parindent=0pt
\psetnum{1}
\course{8.311}
\rsection{1}
\name{Yichao Yu}
\renewcommand{\thesection}{\arabic{section}.}
\renewcommand{\thesubsection}{(\alph{subsection})}
\renewcommand{\thesubsubsection}{\roman{subsubsection}.}

\begin{document}
\section{}
\subsection{}
\eqar{
  &\paren{\vec A\times\paren{\vec B\times\vec C}}_i\\
  =&\varepsilon_{ijk}A_j\paren{\vec B\times\vec C}_k\\
  =&\varepsilon_{kij}\varepsilon_{klm}A_jB_lC_m\\
  =&\paren{\delta_{il}\delta_{jm}-\delta_{im}\delta_{jl}}A_jB_lC_m\\
  =&B_iA_jC_j-C_iA_jB_j\\
  =&\paren{\vec B\paren{\vec A\cdot\vec C}-\vec C\paren{\vec A\cdot\vec B}}_i
}

\subsection{}
\eqar{
  &\vec A\times\paren{\nabla\times\vec A}\\
  =&\paren{\nabla\otimes\vec A}\cdot\vec A-\paren{\vec A\cdot\nabla}\vec A\\
  =&\frac12\nabla\paren{A^2}-\paren{\vec A\cdot\nabla}\vec A
}

\subsection{}
Use $X_c$ to respresent treating $X$ as constant during the derivative.
\eqar{
  &\nabla\times(\vec A\times\vec B)\\
  =&\nabla\times(\vec A_c\times\vec B)+\nabla\times(\vec A\times\vec B_c)\\
  =&\vec A_c(\nabla\cdot\vec B)-(\vec A_c\cdot\nabla)\vec B-\vec B_c(\nabla\cdot\vec A)+(\vec B_c\cdot\nabla)\vec A\\
  =&\vec A(\nabla\cdot\vec B)-(\vec A\cdot\nabla)\vec B-\vec B(\nabla\cdot\vec A)+(\vec B\cdot\nabla)\vec A
}

\section{}
\subsection{}
\eqar{
  &\int^{max}_{min} f(x)\Theta'(x-a)\ud x\\
  =&\int^{max}_{min} f(x)\ud\Theta(x-a)\\
  =&\left.f(x)\Theta(x-a)\right|^{max}_{min}-\int^{max}_{min}\Theta(x-a)\ud f(x)\\
  =&f(max)-\int^{max}_{a}\ud f(x)\\
  =&f(a)
}
\subsection{}
\eqar{
  \diff{\mathbf{sgn}(t)}{t}=&\diff{2\Theta(t)-1}{t}\\
  =&2\delta(t)
}
\subsection{}
\eqar{
  \rho\paren{r, \theta, \phi}=&\frac{Q}{4\pi R^2}\delta\paren{r-R}
}
\subsection{}
\eqar{
  \rho\paren{\rho, \theta, z}=&\frac{\lambda}{2\pi b}\delta\paren{\rho-b}
}
\subsection{}
Assuming the disk is parallel to the $x-y$ plane at $z_0$.
\eqar{
  \rho\paren{\rho, \theta, z}=&\frac{Q}{\pi b^2}\delta\paren{z-z_0}\Theta\paren{b-r}
}

\section{}
\subsection{}
\eqar{
  g_a=&\nabla^2f_a\\
  =&-\frac1{4\pi r^2}\pdiff{}{r}\paren{r^2\pdiff{}{r}\paren{\frac{1}{\sqrt{r^2+a^2}}}}\\
  =&\frac1{4\pi r^2}\pdiff{}{r}\paren{\frac{r^3}{\sqrt{r^2+a^2}^3}}\\
  =&\frac{3}{4\pi\paren{r^2+a^2}}\pdiff{}{r}\paren{\frac{r}{\sqrt{r^2+a^2}}}\\
  =&\frac{3a^2}{4\pi\paren{r^2+a^2}^{5/2}}
}
\begin{tikzpicture}[thick]
  \draw[->] (0, 0) -- (14, 0) node[right] {$r$};
  \draw[->] (0, 0) -- (0, 7) node[above] {$g$};
  \draw[red] plot[samples=2000, domain=0:13]
  function {3 * 0.2**2 / 4 / pi / ((x / 16)**2 + 0.2**2)**(5 / 2)};
  \draw[green] plot[samples=2000, domain=0:13]
  function {3 * 0.3**2 / 4 / pi / ((x / 16)**2 + 0.3**2)**(5 / 2)};
  \draw[blue] plot[samples=2000, domain=0:13]
  function {3 * 0.4**2 / 4 / pi / ((x / 16)**2 + 0.4**2)**(5 / 2)};
\end{tikzpicture}
\subsection{}
\eqar{
  \int\ud r 4\pi r^2 g_a=&\int\ud r 4\pi r^2 \frac{3a^2}{4\pi\paren{r^2+a^2}^{5/2}}\\
  =&\int\frac{3\rho^2\ud\rho}{\paren{1+\rho^2}^{5/2}}\\
  =&\int_0^{\pi/2}3\sin^2\theta\ud\sin\theta\\
  =&1
}
\subsection{}
\eqar{
  \lim_{a\rightarrow0,r\neq0}g_a\paren{r}=&\lim_{a\rightarrow0,r\neq0}\frac{3a^2}{4\pi\paren{r^2+a^2}^{5/2}}\\
  =&\lim_{a\rightarrow0,r\neq0}\frac{3a^2}{4\pir^5}\\
  =&0
}

\end{document}
