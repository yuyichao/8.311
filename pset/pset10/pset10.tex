\documentclass[10pt,fleqn]{article}
\newcommand{\name}[1]{\def\psettitlename{#1}}
\newcommand{\course}[1]{\def\psettitlecourse{#1}}
\newcommand{\rsection}[1]{\def\psettitlersection{#1}}
\newcommand{\psetnum}[1]{\def\psettitlepsetnum{#1}}
% \usepackage[journal=rsc]{chemstyle}
% \usepackage{mhchem}
\usepackage{amsmath}
\usepackage{amssymb}
\usepackage{amsfonts}
\usepackage{esint}
\usepackage{bbm}
\usepackage{amscd}
\usepackage{picinpar}
\usepackage[pdftex]{graphicx}
\usepackage{tikz}
\usepackage{indentfirst}
\usepackage{wrapfig}
\usepackage{units}
\usepackage{textcomp}
\usepackage[utf8x]{inputenc}
% \usepackage{feyn}
\usepackage{feynmp}
\usetikzlibrary{
  arrows,
  calc,
  decorations.pathmorphing,
  decorations.pathreplacing,
  decorations.markings,
  fadings,
  positioning,
  shapes
}

\DeclareGraphicsRule{*}{mps}{*}{}
\newcommand{\ud}{\mathrm{d}}
\newcommand{\ue}{\mathrm{e}}
\newcommand{\ui}{\mathrm{i}}
\newcommand{\res}{\mathrm{Res}}
\newcommand{\Tr}{\mathrm{Tr}}
\newcommand{\dsum}{\displaystyle\sum}
\newcommand{\dprod}{\displaystyle\prod}
\newcommand{\dlim}{\displaystyle\lim}
\newcommand{\dint}{\displaystyle\int}
\newcommand{\fsno}[1]{{\!\not\!{#1}}}
\newcommand{\eqar}[1]
{
  \begin{align*}
    #1
  \end{align*}
}
\newcommand{\texp}[2]{\ensuremath{{#1}\times10^{#2}}}
\newcommand{\dexp}[2]{\ensuremath{{#1}\cdot10^{#2}}}
\newcommand{\eval}[2]{{\left.{#1}\right|_{#2}}}
\newcommand{\paren}[1]{{\left({#1}\right)}}
\newcommand{\lparen}[1]{{\left({#1}\right.}}
\newcommand{\rparen}[1]{{\left.{#1}\right)}}
\newcommand{\abs}[1]{{\left|{#1}\right|}}
\newcommand{\sqr}[1]{{\left[{#1}\right]}}
\newcommand{\crly}[1]{{\left\{{#1}\right\}}}
\newcommand{\angl}[1]{{\left\langle{#1}\right\rangle}}
\newcommand{\tpdiff}[4][{}]{{\paren{\frac{\partial^{#1} {#2}}{\partial {#3}{}^{#1}}}_{#4}}}
\newcommand{\tpsdiff}[4][{}]{{\paren{\frac{\partial^{#1}}{\partial {#3}{}^{#1}}{#2}}_{#4}}}
\newcommand{\pdiff}[3][{}]{{\frac{\partial^{#1} {#2}}{\partial {#3}{}^{#1}}}}
\newcommand{\diff}[3][{}]{{\frac{\ud^{#1} {#2}}{\ud {#3}{}^{#1}}}}
\newcommand{\psdiff}[3][{}]{{\frac{\partial^{#1}}{\partial {#3}{}^{#1}} {#2}}}
\newcommand{\sdiff}[3][{}]{{\frac{\ud^{#1}}{\ud {#3}{}^{#1}} {#2}}}
\newcommand{\tpddiff}[4][{}]{{\left(\dfrac{\partial^{#1} {#2}}{\partial {#3}{}^{#1}}\right)_{#4}}}
\newcommand{\tpsddiff}[4][{}]{{\paren{\dfrac{\partial^{#1}}{\partial {#3}{}^{#1}}{#2}}_{#4}}}
\newcommand{\pddiff}[3][{}]{{\dfrac{\partial^{#1} {#2}}{\partial {#3}{}^{#1}}}}
\newcommand{\ddiff}[3][{}]{{\dfrac{\ud^{#1} {#2}}{\ud {#3}{}^{#1}}}}
\newcommand{\psddiff}[3][{}]{{\frac{\partial^{#1}}{\partial{}^{#1} {#3}} {#2}}}
\newcommand{\sddiff}[3][{}]{{\frac{\ud^{#1}}{\ud {#3}{}^{#1}} {#2}}}
\usepackage{fancyhdr}
\usepackage{multirow}
\usepackage{fontenc}
% \usepackage{tipa}
\usepackage{ulem}
\usepackage{color}
\usepackage{cancel}
\newcommand{\hcancel}[2][black]{\setbox0=\hbox{#2}%
  \rlap{\raisebox{.45\ht0}{\textcolor{#1}{\rule{\wd0}{1pt}}}}#2}
\pagestyle{fancy}
\setlength{\headheight}{67pt}
\fancyhead{}
\fancyfoot{}
\fancyfoot[C]{\thepage}
\fancyhead[R]
{
  \psettitlename \\
  \psettitlecourse{} Problem Set \psettitlepsetnum \\
  \ifx\psettitlersection\empty
  \else
  Recitation Section \psettitlersection
  \fi
}
\renewcommand{\footruleskip}{0pt}
\renewcommand{\headrulewidth}{0.4pt}
\renewcommand{\footrulewidth}{0pt}
\addtolength{\hoffset}{-1.3cm}
\addtolength{\voffset}{-2cm}
\addtolength{\textwidth}{3cm}
\addtolength{\textheight}{2.5cm}
\renewcommand{\footskip}{10pt}
\setlength{\headwidth}{\textwidth}
\setlength{\headsep}{20pt}
\setlength{\marginparwidth}{0pt}
\parindent=0pt
\psetnum{10}
\course{8.311}
\rsection{1}
\name{Yichao Yu}
\renewcommand{\thesection}{\arabic{section}.}
\renewcommand{\thesubsection}{(\alph{subsection})}
\renewcommand{\thesubsubsection}{\roman{subsubsection}.}

\begin{document}
\section{}
\subsection{}
From the generic form
\eqar{
  V=&V_0J_0\paren{\frac{x_{01}\rho}{a}}\frac{\sinh\paren{\dfrac{x_{01}z}{a}}}{\sinh\paren{\dfrac{x_{01}L}{a}}}
}
\subsection{}
\eqar{
  V=&V_0J_0\paren{\frac{\rho}{a}}\exp\paren{-\frac{z}{a}}
}

\section{}
\subsection{}
The flux through a circle defined by $r$ and $\theta$, for $r < R$
\eqar{
  \Phi=&\pi r^2\sin^2\theta B
  \intertext{For $r > R$}
  \Phi=&\int_{\cos\theta}^12\pi r^2\ud z\frac{\mu_0 m}{4\pi r^3}2z\\
  =&\pi\frac{R^3}{r}\sin^2\theta B
  \intertext{$E$ field}
  E_\phi=&\frac{1}{2\pi r\sin\theta}\diff{\Phi}{t}\\
  =&\diff{B}{t}\sin\theta\left\{
    \begin{array}{ll}
      \dfrac{r}{2}&(r < R)\\
      \dfrac{R^3}{2r^2}&(r > R)
    \end{array}
  \right.
}
\subsection{}
\eqar{
  U_{B_{out}}=&\frac{1}{2\mu_0}\int_R^\infty\ud r\int_0^\pi r\ud\theta\int_0^{2\pi}r\sin\theta\ud\phi\frac{\mu_0^2 m^2}{16\pi^2r^6}\paren{4\cos^2\theta+\sin^2\theta}\\
  =&\frac{\mu_0 m^2}{16\pi}\int_R^\infty\ud r\frac{1}{r^4}\int_{-1}^1\ud\cos\theta\paren{4\cos^2\theta+\sin^2\theta}\\
  =&\frac{\mu_0 m^2}{4\pi}\frac{1}{3R^3}\\
  =&\frac{\mu_0 m^2}{12\pi R^3}
  \intertext{Inside}
  U_{B_{in}}=&\frac{4\pi}{3}R^3\frac{B^2}{2\mu_0}\\
  =&\frac{4\pi}{3}R^3\frac{1}{2\mu_0}\frac{4\mu_0^2}{9}\frac{9}{16\pi^2 R^6}m^2\\
  =&\frac{\mu_0m^2}{6\pi R^3}
  \intertext{Total}
  U_{B}=&\frac{\mu_0m^2}{4\pi R^3}
}
\subsection{}
\eqar{
  W=&\int_0^{\pi} R\ud\theta\int_0^{2\pi} R\sin\theta\ud\phi\kappa\sin\theta\diff{B}{t}\sin\theta\frac R2\\
  =&\pi \frac{\mu_0}{3} R^3 \diff{\kappa^2}{t}\int_{-1}^1\ud z \paren{1-z^2}\\
  =&\frac{\mu_0}{4\pi R^3}\diff{m^2}{t}\\
  =&\diff{U_B}{t}
}
\subsection{}
For $r=R+0^+$
\eqar{
  W_+=&\int_0^{\pi} R\ud\theta\int_0^{2\pi} R\sin\theta\ud\phi
  \frac{\mu_0m}{4\pi R^3}\sin\theta\frac{1}{\mu_0}\diff{B}{t}\sin\theta\frac R2\\
  =&\int_0^{\pi} R\ud\theta\int_0^{2\pi} R\sin\theta\ud\phi
  \frac{\kappa}{3}\sin\theta\diff{B}{t}\sin\theta\frac R2\\
  =&\frac{W}{3}
  \intertext{For $r=R-0^+$}
  W_-=&\int_0^{\pi} R\ud\theta\int_0^{2\pi} R\sin\theta\ud\phi
  B\sin\theta\frac{1}{\mu_0}\diff{B}{t}\sin\theta\frac R2\\
  =&\int_0^{\pi} R\ud\theta\int_0^{2\pi} R\sin\theta\ud\phi
  \frac{2}{3}\kappa\sin\theta\diff{B}{t}\sin\theta\frac R2\\
  =&\frac{2W}{3}
}
This is the same as what one would expect from (b) and (c) with determines the ratio and the sum of $W_-$ and $W_+$ respectively.
\subsection{}
From the symmetry of the problem, only $z$ component of angular momentum can be non-zero. Since the ``static'' $E$ field is $0$ inside the sphere, we only need to consider the space outside the sphere.
\eqar{
  L_z=&\int_R^\infty\ud r\int_0^\pi r\ud\theta\int_0^{2\pi}r\sin\theta\ud\phi
  \hat z\cdot\paren{\vec r\times\paren{\varepsilon_0 \vec E\times\vec B}}\\
  =&2\pi\varepsilon_0\int_R^\infty\ud r\int_0^\pi \ud\theta r^2\sin\theta
  \paren{\hat r\times\hat\theta}\cdot\paren{\hat z\times\vec r}EB_\theta\\
  =&2\pi\varepsilon_0\int_R^\infty\ud r\int_0^\pi \ud\theta r^2\sin\theta
  \hat\phi\cdot\hat\phi r\sin\theta \frac{Q}{4\pi\varepsilon_0 r^2}\frac{\mu_0 m}{4\pi r^3}\sin\theta\\
  =&\frac{\mu_0 mQ}{8\pi}\int_R^\infty\frac{\ud r}{r^2}
  \int_{-1}^1 \ud z\paren{1-z^2}\\
  =&\frac{\mu_0 mQ}{6\pi R}
}

\subsection{}
Torque,
\eqar{
  \tau_z=&\int_0^\pi R\ud\theta\int_0^{2\pi}R\sin\theta\ud\phi
  \hat z\cdot\paren{\vec R\times\paren{\sigma\vec E+\vec \kappa\times\vec B}}\\
  =&2\pi\int_0^\pi\ud\theta R^2\sin\theta
  \paren{\sigma\vec E+\vec \kappa\times\vec B}\cdot\paren{\hat z\times\vec R}\\
  =&2\pi\int_0^\pi\ud\theta R^2\sin\theta
  \paren{\sigma\vec E\cdot\hat\phi+\paren{\vec \kappa\times\vec B}\cdot\hat\phi} R\sin\theta\\
  =&2\pi\int_0^\pi\ud\theta R^3\sin^2\theta
  \paren{\sigma E_\phi+\paren{\hat\phi\times\vec \kappa}\cdot\vec B}\\
  =&\pi R^4\diff{B}{t}\sigma\int_0^\pi\ud\theta\sin^3\theta\\
  =&\pi R^4\diff{m}{t}\frac{3}{4\pi R^3}\frac{2\mu_0}{3}\frac{Q}{4\pi R^2}
  \int_{-1}^1\ud z\paren{1-z^2}\\
  =&\frac{\mu_0Q}{6\pi R}\diff{m}{t}
}
\subsection{}
\eqar{
  \tau_z=&\int_0^\pi R\ud\theta\int_0^{2\pi}R\sin\theta\ud\phi
  \hat z\cdot
  \paren{\vec R\times \paren{\varepsilon_0 \vec E\vec E+\frac1{\mu_0}\vec B\vec B-\frac{\varepsilon}{2}E^2+\frac1{2\mu_0}B^2}}\cdot\hat R\\
  =&\int_0^\pi R\ud\theta\int_0^{2\pi}R\sin\theta\ud\phi
  \hat z\cdot
  \paren{\vec R\times \paren{\varepsilon_0 \vec E\paren{\vec E\cdot\hat R}+\frac1{\mu_0}\vec B\paren{\vec B\cdot\hat R}}}\\
  =&\int_0^\pi R\ud\theta\int_0^{2\pi}R\sin\theta\ud\phi
  \paren{\varepsilon_0 \vec E E_r+\frac1{\mu_0}\vec B B_r}\cdot
  \paren{\hat z\times\vec R}\\
  =&2\pi R^3\int_0^\pi \ud\theta\sin^2\theta
  \paren{\varepsilon_0 \vec E E_r+\frac1{\mu_0}\vec B B_r}\cdot\hat \phi\\
  =&2\pi \varepsilon_0 R^3\int_0^\pi \ud\theta\sin^2\theta E_\phi E_r
  \intertext{Since $E_r=0$ for $r<R$, the flux is only non-zero outside the shell. For $r=R+0^+$}
  \tau_z=&2\pi \varepsilon_0 R^3 \frac{Q}{4\pi\varepsilon_0 R^2} \int_0^\pi \ud\theta\sin^2\theta \diff{B}{t}\sin\theta\frac R2\\
  =&\frac{QR^2}{4} \diff{B}{t} \int_0^\pi \ud\theta\sin^3\theta\\
  =&\frac{QR^2}{4} \diff{m}{t}\frac{3}{4\pi R^3}\frac{2\mu_0}{3} \frac{4}{3}\\
  =&\frac{\mu_0Q}{6\pi R}\diff{m}{t}
}

\end{document}
