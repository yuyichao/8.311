\documentclass[10pt,fleqn]{article}
\newcommand{\name}[1]{\def\psettitlename{#1}}
\newcommand{\course}[1]{\def\psettitlecourse{#1}}
\newcommand{\rsection}[1]{\def\psettitlersection{#1}}
\newcommand{\psetnum}[1]{\def\psettitlepsetnum{#1}}
%\usepackage[journal=rsc]{chemstyle}
%\usepackage{mhchem}
\usepackage{amsmath}
\usepackage{amssymb}
\usepackage{amsfonts}
\usepackage{esint}
\usepackage{bbm}
\usepackage{amscd}
\usepackage{picinpar}
\usepackage[pdftex]{graphicx}
\usepackage{tikz}
\usepackage{indentfirst}
\usepackage{wrapfig}
\usepackage{units}
\usepackage{textcomp}
\usepackage[utf8x]{inputenc}
% \usepackage{feyn}
\usepackage{feynmp}
\usetikzlibrary{
  arrows,
  calc,
  decorations.pathmorphing,
  decorations.pathreplacing,
  decorations.markings,
  fadings,
  positioning,
  shapes
}

\DeclareGraphicsRule{*}{mps}{*}{}
\newcommand{\ud}{\mathrm{d}}
\newcommand{\ue}{\mathrm{e}}
\newcommand{\ui}{\mathrm{i}}
\newcommand{\res}{\mathrm{Res}}
\newcommand{\Tr}{\mathrm{Tr}}
\newcommand{\dsum}{\displaystyle\sum}
\newcommand{\dprod}{\displaystyle\prod}
\newcommand{\dlim}{\displaystyle\lim}
\newcommand{\dint}{\displaystyle\int}
\newcommand{\fsno}[1]{{\!\not\!{#1}}}
\newcommand{\eqar}[1]
{
  \begin{align*}
    #1
  \end{align*}
}
\newcommand{\texp}[2]{\ensuremath{{#1}\times10^{#2}}}
\newcommand{\dexp}[2]{\ensuremath{{#1}\cdot10^{#2}}}
\newcommand{\eval}[2]{{\left.{#1}\right|_{#2}}}
\newcommand{\paren}[1]{{\left({#1}\right)}}
\newcommand{\lparen}[1]{{\left({#1}\right.}}
\newcommand{\rparen}[1]{{\left.{#1}\right)}}
\newcommand{\abs}[1]{{\left|{#1}\right|}}
\newcommand{\sqr}[1]{{\left[{#1}\right]}}
\newcommand{\crly}[1]{{\left\{{#1}\right\}}}
\newcommand{\angl}[1]{{\left\langle{#1}\right\rangle}}
\newcommand{\tpdiff}[4][{}]{{\paren{\frac{\partial^{#1} {#2}}{\partial {#3}{}^{#1}}}_{#4}}}
\newcommand{\tpsdiff}[4][{}]{{\paren{\frac{\partial^{#1}}{\partial {#3}{}^{#1}}{#2}}_{#4}}}
\newcommand{\pdiff}[3][{}]{{\frac{\partial^{#1} {#2}}{\partial {#3}{}^{#1}}}}
\newcommand{\diff}[3][{}]{{\frac{\ud^{#1} {#2}}{\ud {#3}{}^{#1}}}}
\newcommand{\psdiff}[3][{}]{{\frac{\partial^{#1}}{\partial {#3}{}^{#1}} {#2}}}
\newcommand{\sdiff}[3][{}]{{\frac{\ud^{#1}}{\ud {#3}{}^{#1}} {#2}}}
\newcommand{\tpddiff}[4][{}]{{\left(\dfrac{\partial^{#1} {#2}}{\partial {#3}{}^{#1}}\right)_{#4}}}
\newcommand{\tpsddiff}[4][{}]{{\paren{\dfrac{\partial^{#1}}{\partial {#3}{}^{#1}}{#2}}_{#4}}}
\newcommand{\pddiff}[3][{}]{{\dfrac{\partial^{#1} {#2}}{\partial {#3}{}^{#1}}}}
\newcommand{\ddiff}[3][{}]{{\dfrac{\ud^{#1} {#2}}{\ud {#3}{}^{#1}}}}
\newcommand{\psddiff}[3][{}]{{\frac{\partial^{#1}}{\partial{}^{#1} {#3}} {#2}}}
\newcommand{\sddiff}[3][{}]{{\frac{\ud^{#1}}{\ud {#3}{}^{#1}} {#2}}}
\usepackage{fancyhdr}
\usepackage{multirow}
\usepackage{fontenc}
%\usepackage{tipa}
\usepackage{ulem}
\usepackage{color}
\usepackage{cancel}
\newcommand{\hcancel}[2][black]{\setbox0=\hbox{#2}%
\rlap{\raisebox{.45\ht0}{\textcolor{#1}{\rule{\wd0}{1pt}}}}#2}
\pagestyle{fancy}
\setlength{\headheight}{67pt}
\fancyhead{}
\fancyfoot{}
\fancyfoot[C]{\thepage}
\fancyhead[R]
{
\psettitlename \\
\psettitlecourse{} Problem Set \psettitlepsetnum \\
\ifx\psettitlersection\empty
\else
Recitation Section \psettitlersection
\fi
}
\renewcommand{\footruleskip}{0pt}
\renewcommand{\headrulewidth}{0.4pt}
\renewcommand{\footrulewidth}{0pt}
\addtolength{\hoffset}{-1.3cm}
\addtolength{\voffset}{-2cm}
\addtolength{\textwidth}{3cm}
\addtolength{\textheight}{2.5cm}
\renewcommand{\footskip}{10pt}
\setlength{\headwidth}{\textwidth}
\setlength{\headsep}{20pt}
\setlength{\marginparwidth}{0pt}
\parindent=0pt
\psetnum{5}
\course{8.311}
\rsection{1}
\name{Yichao Yu}
\renewcommand{\thesection}{\arabic{section}.}
\renewcommand{\thesubsection}{(\alph{subsection})}
\renewcommand{\thesubsubsection}{\roman{subsubsection}.}

\begin{document}
\section{}
\subsection{}
\eqar{
  \diff{W_{rad}}{t'}=&\frac{e^2}{4\pi\varepsilon_0}\frac{2}{3c}\gamma^6\paren{\Omega^2\beta^2-\Omega^2\beta^4}\\
  =&\frac{e^2}{4\pi\varepsilon_0}\frac{2}{3c}\gamma^4\Omega^2\beta^2\\
  =&\frac{e^2}{4\pi\varepsilon_0}\frac{2\omega_0^2}{3c}\gamma^2\beta^2
}
\subsection{}
\eqar{
  \diff{\gamma}{t}mc^2=&-\frac{e^2}{4\pi\varepsilon_0}\frac{2\omega_0^2}{3c}\gamma^2\beta^2\\
  \diff{\gamma}{t}=&-\frac{e^2}{4\pi\varepsilon_0}\frac{2\omega_0^2}{3mc^3}\gamma^2\beta^2\\
  =&-\frac{2\omega_0^2r_e}{3c}\gamma^2\beta^2\\
  T_0=&\frac{3c}{2\omega_0^2r_e}
}
\subsection{}
For $\gamma\gg1$, $\beta\approx1$
\eqar{
  \diff{\gamma}{t}=&-\frac{\gamma^2}{T_0}\\
  \frac{1}{\gamma}=&\frac{1}{\gamma_0}+\frac{t}{T_0}\\
  T=&\frac{\gamma_0 - \gamma}{\gamma\gamma_0}T_0
}
\subsection{}
\eqar{
  \omega_{break}=&3\gamma_e^2\omega_0\\
  \gamma_e=&\sqrt{\frac{\omega_{break}}{3\omega_0}}
}
\subsection{}
\eqar{
  T=&\frac{T_0}{\gamma}\\
  =&T_0\sqrt{\frac{3\omega_0}{\omega_{break}}}
}
\subsection{}
\eqar{
  \omega_0=&\frac{eB}{m}\\
  =&1.76\cdot10^3\\
  T=&\frac{3c}{2\omega_0^2r_e}\sqrt{\frac{3\omega_0}{\omega_{break}}}\\
  =&5.16\cdot10^{10}s\\
  =&1635\text{a}
}

\section{}
\subsection{}
\eqar{
  E_r=&\left\{
    \begin{array}{ll}
      \dfrac{2p_0\cos\theta}{4\pi\varepsilon_0r^3}&\paren{r>R}\\
      E_0\cos\theta&\paren{r<R}
    \end{array}
  \right.\\
  E_\theta=&\left\{
    \begin{array}{ll}
      \dfrac{p_0\sin\theta}{4\pi\varepsilon_0r^3}&\paren{r>R}\\
      -E_0\sin\theta&\paren{r<R}
    \end{array}
  \right.\\
  E_0=&-\dfrac{p_0}{4\pi\varepsilon_0R^3}\\
  \frac{\sigma_0}{\varepsilon_0}=&\dfrac{2p_0}{4\pi\varepsilon_0R^3}-E_0\\
  p_0=&\frac{4\pi R^3\sigma_0}{3}\\
  E_0=&\frac{\sigma_0}{3\varepsilon_0}
}
\subsection{}
\eqar{
  B_r=&\left\{
    \begin{array}{ll}
      \dfrac{2\mu_0m_0\cos\theta}{4\pi r^3}&\paren{r>R}\\
      B_0\cos\theta&\paren{r<R}
    \end{array}
  \right.\\
  B_\theta=&\left\{
    \begin{array}{ll}
      \dfrac{\mu_0m_0\sin\theta}{4\pi r^3}&\paren{r>R}\\
      -B_0\sin\theta&\paren{r<R}
    \end{array}
  \right.\\
  B_0=&\dfrac{\mu_0m_0}{2\pi R^3}\\
  \mu_0\kappa_0=&\dfrac{\mu_0m_0}{4\pi R^3}+B_0\\
  m_0=&\frac{4\pi R^3\kappa_0}{3}\\
  B_0=&\frac{2\mu_0\kappa_0}{3}
}

\section{}
\subsection{}
\eqar{
  E_i=&-\frac{p_j}{4\pi\varepsilon_0}\partial_i\frac{r_j}{r^3}\\
  =&\frac{p_j}{4\pi\varepsilon_0}\partial_i\partial_j\frac{1}{r}
  \intertext{Since $\nabla^2\dfrac{1}{r}=-4\pi\delta\paren{\vec r}$}
  \partial_i\frac{r_j}{r^3}=&\paren{\frac{\delta_{ij}}{r^3}-3\frac{x_ix_j}{r^5}}+\frac{4\pi}{3}\delta_{ij}\delta\paren{\vec r}\\
  E_i=&\frac{1}{4\pi\varepsilon_0}\paren{3\frac{x_ix_jp_j}{r^5}-\frac{p_i}{r^3}}-\frac{1}{3\varepsilon_0}p_i\delta\paren{\vec r}
}
\subsection{}
\eqar{
  \nabla\times\paren{\frac{\mu_0}{4\pi}\frac{\vec m\times\vec r}{r^3}}=&\frac{\mu_0}{4\pi}\paren{\vec m\paren{\nabla\cdot\frac{\vec r}{r^3}}-\vec m\cdot\paren{\nabla\frac{\vec r}{r^3}}}\\
  =&\frac{\mu_0}{4\pi}\paren{\vec m 4\pi\delta\paren{\vec r}+
      3\frac{\paren{\vec m\cdot\vec r}\vec r}{r^5}-\frac{\vec m}{r^3}-\frac{4\pi\vec m}{3}\delta\paren{\vec r}
  }\\
  =&\frac{\mu_0}{4\pi}\paren{3\frac{\paren{\vec m\cdot\vec r}\vec r}{r^5}-\frac{\vec m}{r^3}}+\frac{2\mu_0\vec m}{3}\delta\paren{\vec r}\\
}
\subsection{}
For electric field the integral of the field inside the sphere is
\eqar{
  \int E=&-\frac{\vec p_0}{3\varepsilon_0}
}
Therefore as $R\rightarrow0$ the field becomes
\eqar{
  E=&-\delta\paren{\vec r}\frac{\vec p_0}{3\varepsilon_0}
}
For magnetic field the integral of the field inside the sphere is
\eqar{
  \int B=&\frac{2\mu_0\vec m_0}{3}
}
Therefore as $R\rightarrow0$ the field becomes
\eqar{
  B=&\delta\paren{\vec r}\frac{2\mu_0\vec m_0}{3}
}

\subsection{}
By symmetry, the field is along $\vec p$ for electric field
\eqar{
  \overline{\vec E}=&\frac{3}{4\pi a^3}\frac{p^2}{4\pi\varepsilon_0}\int_0^{2\pi}\ud\phi\int_0^{\pi}\sin\theta\ud\theta\int_0^a\ud r\frac{3\cos^2\theta-1}{r^3}\\
  =&\frac{3}{4\pi a^3}\frac{p^2}{2\varepsilon_0}\int^1_{-1}\ud\cos\theta\int_0^a\ud r\frac{3\cos^2\theta-1}{r^3}\\
  =&0
}
Similarly
\eqar{
  \overline{\vec B}=&0
}

\subsection{}
Since the average field generated by the normal part is $0$, the sign of the field is determined by the $\delta$ part.

\section{}
\subsection{}
\eqar{
  \vec E\paren{x, t}=&-\hat y\frac{c\mu_0}{2}\int_{-\infty}^{\infty}\ud x'J_0\exp\paren{\ui\paren{kx'-\omega\paren{t-\frac{\abs{x-x'}}c}}}
}
\subsection{}
\eqar{
  \vec E\paren{x, t}=&-\hat y\frac{c\mu_0J_0\ue^{\ui\paren{kx-\omega t}}}{2}\paren{\int_{-\infty}^{0}\ud x''\exp\paren{\ui\paren{k-\frac{\omega}c}x''}+\int_{0}^{\infty}\ud x''\exp\paren{\ui\paren{k+\frac{\omega}c}x''}}\\
  =&-\hat y\frac{c\mu_0J_0\ue^{\ui\paren{kx-\omega t}}}{2\ui}\paren{\frac{1}{k-\omega/c}-\frac{1}{k+\omega/c}}\\
  =&\ui\hat yc\mu_0J_0\ue^{\ui\paren{kx-\omega t}}\frac{\omega/c}{k^2-\omega^2/c^2}
}
\subsection{}
\eqar{
  J_{pol}=&\varepsilon_0\paren{k_e-1}\omega\hat yc\mu_0J_0\ue^{\ui\paren{kx-\omega t}}\frac{\omega/c}{k^2-\omega^2/c^2}\\
  =&J_0\hat y\ue^{\ui\paren{kx-\omega t}}\\
  1=&\varepsilon_0\paren{k_e-1}\omega c\mu_0\frac{\omega/c}{k^2-\omega^2/c^2}
  \intertext{Let $v\equiv\frac{\omega}{k}$}
  1=&\paren{k_e-1}\frac{v^2}{c^2-v^2}\\
  \frac{c^2}{v^2}=&k_e\\
  v=&\frac{c}{\sqrt{k_e}}
}

\end{document}
