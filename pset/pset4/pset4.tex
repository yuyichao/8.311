\documentclass[10pt,fleqn]{article}
\newcommand{\name}[1]{\def\psettitlename{#1}}
\newcommand{\course}[1]{\def\psettitlecourse{#1}}
\newcommand{\rsection}[1]{\def\psettitlersection{#1}}
\newcommand{\psetnum}[1]{\def\psettitlepsetnum{#1}}
%\usepackage[journal=rsc]{chemstyle}
%\usepackage{mhchem}
\usepackage{amsmath}
\usepackage{amssymb}
\usepackage{amsfonts}
\usepackage{esint}
\usepackage{bbm}
\usepackage{amscd}
\usepackage{picinpar}
\usepackage[pdftex]{graphicx}
\usepackage{tikz}
\usepackage{indentfirst}
\usepackage{wrapfig}
\usepackage{units}
\usepackage{textcomp}
\usepackage[utf8x]{inputenc}
% \usepackage{feyn}
\usepackage{feynmp}
\usetikzlibrary{
  arrows,
  calc,
  decorations.pathmorphing,
  decorations.pathreplacing,
  decorations.markings,
  fadings,
  positioning,
  shapes
}

\DeclareGraphicsRule{*}{mps}{*}{}
\newcommand{\ud}{\mathrm{d}}
\newcommand{\ue}{\mathrm{e}}
\newcommand{\ui}{\mathrm{i}}
\newcommand{\res}{\mathrm{Res}}
\newcommand{\Tr}{\mathrm{Tr}}
\newcommand{\dsum}{\displaystyle\sum}
\newcommand{\dprod}{\displaystyle\prod}
\newcommand{\dlim}{\displaystyle\lim}
\newcommand{\dint}{\displaystyle\int}
\newcommand{\fsno}[1]{{\!\not\!{#1}}}
\newcommand{\eqar}[1]
{
  \begin{align*}
    #1
  \end{align*}
}
\newcommand{\texp}[2]{\ensuremath{{#1}\times10^{#2}}}
\newcommand{\dexp}[2]{\ensuremath{{#1}\cdot10^{#2}}}
\newcommand{\eval}[2]{{\left.{#1}\right|_{#2}}}
\newcommand{\paren}[1]{{\left({#1}\right)}}
\newcommand{\lparen}[1]{{\left({#1}\right.}}
\newcommand{\rparen}[1]{{\left.{#1}\right)}}
\newcommand{\abs}[1]{{\left|{#1}\right|}}
\newcommand{\sqr}[1]{{\left[{#1}\right]}}
\newcommand{\crly}[1]{{\left\{{#1}\right\}}}
\newcommand{\angl}[1]{{\left\langle{#1}\right\rangle}}
\newcommand{\tpdiff}[4][{}]{{\paren{\frac{\partial^{#1} {#2}}{\partial {#3}{}^{#1}}}_{#4}}}
\newcommand{\tpsdiff}[4][{}]{{\paren{\frac{\partial^{#1}}{\partial {#3}{}^{#1}}{#2}}_{#4}}}
\newcommand{\pdiff}[3][{}]{{\frac{\partial^{#1} {#2}}{\partial {#3}{}^{#1}}}}
\newcommand{\diff}[3][{}]{{\frac{\ud^{#1} {#2}}{\ud {#3}{}^{#1}}}}
\newcommand{\psdiff}[3][{}]{{\frac{\partial^{#1}}{\partial {#3}{}^{#1}} {#2}}}
\newcommand{\sdiff}[3][{}]{{\frac{\ud^{#1}}{\ud {#3}{}^{#1}} {#2}}}
\newcommand{\tpddiff}[4][{}]{{\left(\dfrac{\partial^{#1} {#2}}{\partial {#3}{}^{#1}}\right)_{#4}}}
\newcommand{\tpsddiff}[4][{}]{{\paren{\dfrac{\partial^{#1}}{\partial {#3}{}^{#1}}{#2}}_{#4}}}
\newcommand{\pddiff}[3][{}]{{\dfrac{\partial^{#1} {#2}}{\partial {#3}{}^{#1}}}}
\newcommand{\ddiff}[3][{}]{{\dfrac{\ud^{#1} {#2}}{\ud {#3}{}^{#1}}}}
\newcommand{\psddiff}[3][{}]{{\frac{\partial^{#1}}{\partial{}^{#1} {#3}} {#2}}}
\newcommand{\sddiff}[3][{}]{{\frac{\ud^{#1}}{\ud {#3}{}^{#1}} {#2}}}
\usepackage{fancyhdr}
\usepackage{multirow}
\usepackage{fontenc}
%\usepackage{tipa}
\usepackage{ulem}
\usepackage{color}
\usepackage{cancel}
\newcommand{\hcancel}[2][black]{\setbox0=\hbox{#2}%
\rlap{\raisebox{.45\ht0}{\textcolor{#1}{\rule{\wd0}{1pt}}}}#2}
\pagestyle{fancy}
\setlength{\headheight}{67pt}
\fancyhead{}
\fancyfoot{}
\fancyfoot[C]{\thepage}
\fancyhead[R]
{
\psettitlename \\
\psettitlecourse{} Problem Set \psettitlepsetnum \\
\ifx\psettitlersection\empty
\else
Recitation Section \psettitlersection
\fi
}
\renewcommand{\footruleskip}{0pt}
\renewcommand{\headrulewidth}{0.4pt}
\renewcommand{\footrulewidth}{0pt}
\addtolength{\hoffset}{-1.3cm}
\addtolength{\voffset}{-2cm}
\addtolength{\textwidth}{3cm}
\addtolength{\textheight}{2.5cm}
\renewcommand{\footskip}{10pt}
\setlength{\headwidth}{\textwidth}
\setlength{\headsep}{20pt}
\setlength{\marginparwidth}{0pt}
\parindent=0pt
\psetnum{4}
\input{../head}
\begin{document}
\section{}
\subsection{}
In $x$, $y$, $z$ basis
\eqar{
  Q_{ij}=&\begin{pmatrix}
    -2d^2q&&\\
    &-2d^2q&\\
    &&4d^2q
  \end{pmatrix}\\
  =&2d^2q\begin{pmatrix}
    -1&&\\
    &-1&\\
    &&2
  \end{pmatrix}
  \intertext{Potential}
  \Phi=&\frac{1}{8\pi\varepsilon_0}\frac{Q_{ij}x_ix_j}{r^5}\\
  =&\frac{d^2q}{4\pi\varepsilon_0}\frac{-x^2-y^2+2z^2}{r^5}\\
  =&\frac{d^2q}{4\pi\varepsilon_0}\frac{3\cos^2\theta-1}{r^3}
  \intertext{Electric field}
  E_r=&\frac{3d^2q}{4\pi\varepsilon_0}\frac{3\cos^2\theta-1}{r^4}\\
  E_\theta=&\frac{d^2q}{4\pi\varepsilon_0}\frac{3\sin2\theta}{r^4}
  \intertext{Field line}
  \diff{r}{\theta}=&r\frac{3\cos^2\theta-1}{\sin2\theta}\\
  \frac{\ud r}{r}=&\frac{3\cos^2\theta-1}{\sin2\theta}\ud\theta\\
  r=&R_0\sin\theta\sqrt{\cos\theta}
  \intertext{The field line is the furthest when $\ddiff{r}{\theta}=0$}
  \theta=\arccos\frac{1}{\sqrt{3}}
}

\subsection{}
Magnetic field
\eqar{
  \vec B=&-\frac{\mu_0}{24\pi c^2 r}\hat r\times\paren{\hat r\cdot\dddot Q}\\
  =&-\frac{\mu_0}{24\pi c^2 r}\hat r\times\paren{\hat r\cdot\diff[3]{2d^2q\paren{3\hat z\hat z-1}}{t}}\\
  =&-\frac{q\mu_0 d_0^2}{12\pi c^2 r}\hat r\times\paren{3\paren{\hat r\cdot\hat z}\hat z-\hat r}\diff[3]{\cos^2\omega t}{t}\\
  =&\frac{q\mu_0 d_0^2\omega^3}{\pi c^2 r}\paren{\hat r\cdot\hat z}\paren{\hat r\times\hat z}\sin2\omega t\\
  =&-\frac{q\mu_0 d_0^2\omega^3}{2\pi c^2 r}\sin2\theta\sin2\omega t\hat e_\phi
  \intertext{Electric field}
  \vec E=&c\vec B\times\vec r\\
  =&-\frac{q\mu_0 d_0^2\omega^3}{2\pi c r}\sin2\theta\sin2\omega t\hat e_\theta
  \intertext{Power}
  \diff{P}{\Omega}=&\frac{q^2d_0^4\omega^6}{8\pi^2\varepsilon_0 c^5}\sin^22\theta
  \intertext{The radiation is the strongest for $\theta=\dfrac\pi4$ and $\theta=\dfrac{3\pi}{4}$}
  P=&\int_0^\pi\ud\theta\sin\theta\int_{0}^{2\pi}\ud\phi\frac{q^2d_0^4\omega^6}{8\pi^2\varepsilon_0 c^5}\sin^22\theta\\
  =&\frac{4q^2d_0^4\omega^6}{15\pi\varepsilon_0 c^5}
}
The radiation appears at $2\omega$
\subsection{}
The radiation will be a dipole radiation appears at $\omega$ with total power
\eqar{
  P=&\frac{q^2d_0^2\omega^4}{12\pi\varepsilon_0 c^3}
  \intertext{Ratio of power}
  \frac{P_{quad}}{P}=&\frac{16d_0^2\omega^2}{5 c^2}\\
  \ll&1
}

\section{}
\subsection{}
\eqar{
  \vec B=&\nabla\times A\\
  =&\frac{\mu_0}{4\pi}\int\ud^3\vec r'\nabla\times\paren{\frac{1}{\abs{\vec r-\vec r'}}\vec j\paren{\vec r', t-\frac{\abs{\vec r-\vec r'}}{c}}}\\
  =&\frac{\mu_0}{4\pi}\int\ud^3\vec r'\frac{1}{\abs{\vec r-\vec r'}}\nabla\times\vec j\paren{\vec r', t-\frac{\abs{\vec r-\vec r'}}{c}}
  +\paren{\nabla\frac{1}{\abs{\vec r-\vec r'}}}\times\vec j\paren{\vec r', t-\frac{\abs{\vec r-\vec r'}}{c}}\\
  =&\frac{\mu_0}{4\pi}\int\ud^3\vec r'\vec j\paren{\vec r', t-\frac{\abs{\vec r-\vec r'}}{c}}\times\frac{\vec r-\vec r'}{\abs{\vec r-\vec r'}^3}
  +\pdiff{}{t}\vec j\paren{\vec r', t-\frac{\abs{\vec r-\vec r'}}{c}}\times\frac{\vec r-\vec r'}{c\abs{\vec r-\vec r'}^2}
}
\subsection{}
The ratio of the two term is on the order of (replacing time derivative with $T$ and replace $\abs{\vec r-\vec r'}$ with $d$)
\eqar{
  \frac{d}{cT}\ll1
}

\section{}
\subsection{}
\eqar{
  P=&\frac{e^4E_0^2}{12\pi\varepsilon_0 m_e^2c^3}\\
  =&\angl{S}\frac{e^4}{6\pi\varepsilon_0^2 m_e^2c^4}\\
  =&\angl{S}\frac{8\pi}{3}r_e^2
}
\subsection{}
The radiation (and therefore the stress tensor) has mirror symmetry along each axis.
\subsection{}
\eqar{
  F=&\frac{P}{c}\\
  =&\angl{S}\frac{8\pi}{3c}r_e^2
}

\section{}
\subsection{}

\section{}

\section{}
\subsection{}
\subsection{}
\subsection{}
\subsection{}
\subsection{}

\end{document}
