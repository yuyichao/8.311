\documentclass[10pt,fleqn]{article}
\newcommand{\name}[1]{\def\psettitlename{#1}}
\newcommand{\course}[1]{\def\psettitlecourse{#1}}
\newcommand{\rsection}[1]{\def\psettitlersection{#1}}
\newcommand{\psetnum}[1]{\def\psettitlepsetnum{#1}}
% \usepackage[journal=rsc]{chemstyle}
% \usepackage{mhchem}
\usepackage{amsmath}
\usepackage{amssymb}
\usepackage{amsfonts}
\usepackage{esint}
\usepackage{bbm}
\usepackage{amscd}
\usepackage{picinpar}
\usepackage[pdftex]{graphicx}
\usepackage{tikz}
\usepackage{indentfirst}
\usepackage{wrapfig}
\usepackage{units}
\usepackage{textcomp}
\usepackage[utf8x]{inputenc}
% \usepackage{feyn}
\usepackage{feynmp}
\usetikzlibrary{
  arrows,
  calc,
  decorations.pathmorphing,
  decorations.pathreplacing,
  decorations.markings,
  fadings,
  positioning,
  shapes
}

\DeclareGraphicsRule{*}{mps}{*}{}
\newcommand{\ud}{\mathrm{d}}
\newcommand{\ue}{\mathrm{e}}
\newcommand{\ui}{\mathrm{i}}
\newcommand{\res}{\mathrm{Res}}
\newcommand{\Tr}{\mathrm{Tr}}
\newcommand{\dsum}{\displaystyle\sum}
\newcommand{\dprod}{\displaystyle\prod}
\newcommand{\dlim}{\displaystyle\lim}
\newcommand{\dint}{\displaystyle\int}
\newcommand{\fsno}[1]{{\!\not\!{#1}}}
\newcommand{\eqar}[1]
{
  \begin{align*}
    #1
  \end{align*}
}
\newcommand{\texp}[2]{\ensuremath{{#1}\times10^{#2}}}
\newcommand{\dexp}[2]{\ensuremath{{#1}\cdot10^{#2}}}
\newcommand{\eval}[2]{{\left.{#1}\right|_{#2}}}
\newcommand{\paren}[1]{{\left({#1}\right)}}
\newcommand{\lparen}[1]{{\left({#1}\right.}}
\newcommand{\rparen}[1]{{\left.{#1}\right)}}
\newcommand{\abs}[1]{{\left|{#1}\right|}}
\newcommand{\sqr}[1]{{\left[{#1}\right]}}
\newcommand{\crly}[1]{{\left\{{#1}\right\}}}
\newcommand{\angl}[1]{{\left\langle{#1}\right\rangle}}
\newcommand{\tpdiff}[4][{}]{{\paren{\frac{\partial^{#1} {#2}}{\partial {#3}{}^{#1}}}_{#4}}}
\newcommand{\tpsdiff}[4][{}]{{\paren{\frac{\partial^{#1}}{\partial {#3}{}^{#1}}{#2}}_{#4}}}
\newcommand{\pdiff}[3][{}]{{\frac{\partial^{#1} {#2}}{\partial {#3}{}^{#1}}}}
\newcommand{\diff}[3][{}]{{\frac{\ud^{#1} {#2}}{\ud {#3}{}^{#1}}}}
\newcommand{\psdiff}[3][{}]{{\frac{\partial^{#1}}{\partial {#3}{}^{#1}} {#2}}}
\newcommand{\sdiff}[3][{}]{{\frac{\ud^{#1}}{\ud {#3}{}^{#1}} {#2}}}
\newcommand{\tpddiff}[4][{}]{{\left(\dfrac{\partial^{#1} {#2}}{\partial {#3}{}^{#1}}\right)_{#4}}}
\newcommand{\tpsddiff}[4][{}]{{\paren{\dfrac{\partial^{#1}}{\partial {#3}{}^{#1}}{#2}}_{#4}}}
\newcommand{\pddiff}[3][{}]{{\dfrac{\partial^{#1} {#2}}{\partial {#3}{}^{#1}}}}
\newcommand{\ddiff}[3][{}]{{\dfrac{\ud^{#1} {#2}}{\ud {#3}{}^{#1}}}}
\newcommand{\psddiff}[3][{}]{{\frac{\partial^{#1}}{\partial{}^{#1} {#3}} {#2}}}
\newcommand{\sddiff}[3][{}]{{\frac{\ud^{#1}}{\ud {#3}{}^{#1}} {#2}}}
\usepackage{fancyhdr}
\usepackage{multirow}
\usepackage{fontenc}
% \usepackage{tipa}
\usepackage{ulem}
\usepackage{color}
\usepackage{cancel}
\newcommand{\hcancel}[2][black]{\setbox0=\hbox{#2}%
  \rlap{\raisebox{.45\ht0}{\textcolor{#1}{\rule{\wd0}{1pt}}}}#2}
\pagestyle{fancy}
\setlength{\headheight}{67pt}
\fancyhead{}
\fancyfoot{}
\fancyfoot[C]{\thepage}
\fancyhead[R]
{
  \psettitlename \\
  \psettitlecourse{} Problem Set \psettitlepsetnum \\
  \ifx\psettitlersection\empty
  \else
  Recitation Section \psettitlersection
  \fi
}
\renewcommand{\footruleskip}{0pt}
\renewcommand{\headrulewidth}{0.4pt}
\renewcommand{\footrulewidth}{0pt}
\addtolength{\hoffset}{-1.3cm}
\addtolength{\voffset}{-2cm}
\addtolength{\textwidth}{3cm}
\addtolength{\textheight}{2.5cm}
\renewcommand{\footskip}{10pt}
\setlength{\headwidth}{\textwidth}
\setlength{\headsep}{20pt}
\setlength{\marginparwidth}{0pt}
\parindent=0pt
\psetnum{9}
\course{8.311}
\rsection{1}
\name{Yichao Yu}
\renewcommand{\thesection}{\arabic{section}.}
\renewcommand{\thesubsection}{(\alph{subsection})}
\renewcommand{\thesubsubsection}{\roman{subsubsection}.}

\begin{document}
\section{}
\subsection{}
\eqar{
  z=&\bar n\int_0^x\frac{\ud x}{\sqrt{n^2-\bar n^2}}\\
  =&\bar n\int_0^x\frac{\ud x}{\sqrt{n_0^2\text{sech}^2\paren{\alpha x}-\bar n^2}}\\
  =&\frac{\cos\theta_0}{\alpha}\int_0^{\alpha x}\frac{\cosh\paren{\alpha x}\ud\alpha x}{\sqrt{1-\cos^2\theta_0\cosh^2\paren{\alpha x}}}\\
  =&\frac{\cos\theta_0}{\alpha}\int_0^{\sinh\paren{\alpha x}}\frac{\ud\sinh\paren{\alpha x}}{\sqrt{\sin^2\theta_0-\cos^2\theta_0\sinh^2\paren{\alpha x}}}\\
  =&\frac{1}{\alpha}\int_0^{\cot\theta_0\sinh\paren{\alpha x}}\frac{\ud y}{\sqrt{1-y^2}}\\
  =&\frac{1}{\alpha}\arcsin\paren{\cot\theta_0\sinh\paren{\alpha x}}\\
  \sin\paren{\alpha z}=&\cot\theta_0\sinh\paren{\alpha x}
  \intertext{Since $\max\paren{\sin\paren{\alpha z}}=1$}
  \sin\paren{\alpha z}=&\frac{\sinh\paren{\alpha x}}{\sinh\paren{\alpha x_{max}}}\\
  \alpha x=&\text{arcsinh}\paren{\sinh\paren{\alpha x_{max}}\sin\paren{\alpha z}}
}
Rays for $\theta_0=\dfrac{\pi}{6},\ \dfrac{\pi}{4},\ \dfrac{\pi}{3}$
\begin{center}
  \begin{tikzpicture}
    \draw[->, line width=1] (-0.5, 0) -- (6.6, 0);
    \draw[red, line width=1] plot[domain=0:2*pi, samples=1000] function {asinh(sin(x) / tan(pi / 6))};
    \draw[red, line width=1] plot[domain=0:2*pi, samples=1000] function {asinh(sin(x) / tan(pi / 4))};
    \draw[red, line width=1] plot[domain=0:2*pi, samples=1000] function {asinh(sin(x) / tan(pi / 3))};
  \end{tikzpicture}
\end{center}

\subsection{}
\eqar{
  Z=&\frac{\pi}{\alpha}
}
Independent from $\bar n$
\subsection{}
\eqar{
  L_{opt}=&\int_{z=0}^Z n\ud s\\
  =&2\int_0^{x_{max}}n\sqrt{1+\paren{\diff{z}{x}}^2}\ud x\\
  =&2\int_0^{x_{max}}n\sqrt{1+\paren{\frac{\bar n}{\sqrt{n^2-\bar n^2}}}^2}\ud x\\
  =&2\int_0^{x_{max}}\frac{n^2}{\sqrt{n^2-\bar n^2}}\ud x\\
  =&2n_0\int_0^{x_{max}}\frac{\text{sech}^2\paren{\alpha x}}{\sqrt{\text{sech}^2\paren{\alpha x}-\cos^2\theta_0}}\ud x\\
  =&\frac{2n_0}{\alpha}\int_0^{\text{sinh}\paren{\alpha x_{max}}}\frac{\ud\text{sinh}\paren{\alpha x}}{\text{cosh}^2\paren{\alpha x}\sqrt{1-\cos^2\theta_0\text{cosh}^2\paren{\alpha x}}}\\
  =&\frac{2n_0}{\alpha}\int_0^{\text{sinh}\paren{\alpha x_{max}}}\frac{\ud y}{\paren{1+y^2}\sqrt{\sin^2\theta_0-\cos^2\theta_0y^2}}\\
  =&\frac{2n_0}{\alpha\cos\theta_0}\frac{\pi}{2\sqrt{1+\tan^2\theta_0}}\\
  =&\frac{\pi n_0}{\alpha}\\
  =&n_0Z
}

\section{}
\subsection{}
For $r<R$, $B_l=0$, for $r>R$, $A_l=0$. Since the charge distribution only have $P_1\paren{\cos\theta}$ component, the only non-zero term in the series is when $l=1$. Therefore, for $r<R$
\eqar{
  \phi_-=&A_1r\cos\theta
  \intertext{And for $r>R$}
  \phi_+=&\frac{B_1}{r^2}\cos\theta
  \intertext{From the boundary condition}
  A_1R=&\frac{B_1}{R^2}\\
  \frac{\sigma}{\varepsilon_0}=&A_1+\frac{2B_1}{R^3}\\
  A_1=&\frac{\sigma}{3\varepsilon_0}\\
  B_1=&\frac{\sigma R^3}{3\varepsilon_0}
}

\subsection{}
For $r<R$, $B_l=0$, for $r>R$, $A_l=0$. Therefore, for $r<R$
\eqar{
  \phi_-=&\sum_l A_lr^lP_l\paren{\cos\theta}\\
  B_{r-}=&\pdiff{\phi_-}{r}\\
  =&\sum_l l A_lr^{l-1}P_l\paren{\cos\theta}\\
  B_{\theta-}=&\frac1r\pdiff{\phi_-}{\theta}\\
  =&\sum_l A_lr^{l-1}\pdiff{P_l\paren{\cos\theta}}{\theta}
  \intertext{And for $r>R$}
  \phi_+=&\sum_l\frac{B_l}{r^{l+1}}P_l\paren{\cos\theta}\\
  B_{r+}=&\pdiff{\phi_+}{r}\\
  =&-\sum_l\frac{\paren{l+1}B_l}{r^{l+2}}P_l\paren{\cos\theta}\\
  B_{\theta+}=&\frac1r\pdiff{\phi_-}{\theta}\\
  =&\sum_l\frac{B_l}{r^{l+2}}\pdiff{P_l\paren{\cos\theta}}{\theta}
  \intertext{From the boundary condition (integrate the relation for $B_\theta$ ignoring a integral/potential constant that doesn't matter)}
  A_l=&-\frac{l+1}{l}\frac{B_l}{R^{2l+1}}\\
  \mu_0\kappa_0\sin\theta=&\sum_l\frac{B_l}{R^{l+2}}\pdiff{P_l\paren{\cos\theta}}{\theta}
  -\sum_l A_lR^{l-1}\pdiff{P_l\paren{\cos\theta}}{\theta}\\
  \mu_0\kappa_0\cos\theta=&\sum_l\paren{\frac{B_l}{R^{l+2}}-A_lR^{l-1}}P_l\paren{\cos\theta}
  \intertext{Therefore only $l=1$ is not vanishing}
  \mu_0\kappa_0=&\frac{B_1}{R^{3}}-A_l\\
  A_1=&-2\frac{B_l}{R^3}\\
  B_1=&\frac{\mu_0\kappa_0R^3}{3}\\
  A_1=&-\frac{2\mu_0\kappa_0}{3}
}

\section{}

\end{document}
