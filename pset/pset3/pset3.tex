\documentclass[10pt,fleqn]{article}
\newcommand{\name}[1]{\def\psettitlename{#1}}
\newcommand{\course}[1]{\def\psettitlecourse{#1}}
\newcommand{\rsection}[1]{\def\psettitlersection{#1}}
\newcommand{\psetnum}[1]{\def\psettitlepsetnum{#1}}
%\usepackage[journal=rsc]{chemstyle}
%\usepackage{mhchem}
\usepackage{amsmath}
\usepackage{amssymb}
\usepackage{amsfonts}
\usepackage{esint}
\usepackage{bbm}
\usepackage{amscd}
\usepackage{picinpar}
\usepackage[pdftex]{graphicx}
\usepackage{tikz}
\usepackage{indentfirst}
\usepackage{wrapfig}
\usepackage{units}
\usepackage{textcomp}
\usepackage[utf8x]{inputenc}
% \usepackage{feyn}
\usepackage{feynmp}
\usetikzlibrary{
  arrows,
  calc,
  decorations.pathmorphing,
  decorations.pathreplacing,
  decorations.markings,
  fadings,
  positioning,
  shapes
}

\DeclareGraphicsRule{*}{mps}{*}{}
\newcommand{\ud}{\mathrm{d}}
\newcommand{\ue}{\mathrm{e}}
\newcommand{\ui}{\mathrm{i}}
\newcommand{\res}{\mathrm{Res}}
\newcommand{\Tr}{\mathrm{Tr}}
\newcommand{\dsum}{\displaystyle\sum}
\newcommand{\dprod}{\displaystyle\prod}
\newcommand{\dlim}{\displaystyle\lim}
\newcommand{\dint}{\displaystyle\int}
\newcommand{\fsno}[1]{{\!\not\!{#1}}}
\newcommand{\eqar}[1]
{
  \begin{align*}
    #1
  \end{align*}
}
\newcommand{\texp}[2]{\ensuremath{{#1}\times10^{#2}}}
\newcommand{\dexp}[2]{\ensuremath{{#1}\cdot10^{#2}}}
\newcommand{\eval}[2]{{\left.{#1}\right|_{#2}}}
\newcommand{\paren}[1]{{\left({#1}\right)}}
\newcommand{\lparen}[1]{{\left({#1}\right.}}
\newcommand{\rparen}[1]{{\left.{#1}\right)}}
\newcommand{\abs}[1]{{\left|{#1}\right|}}
\newcommand{\sqr}[1]{{\left[{#1}\right]}}
\newcommand{\crly}[1]{{\left\{{#1}\right\}}}
\newcommand{\angl}[1]{{\left\langle{#1}\right\rangle}}
\newcommand{\tpdiff}[4][{}]{{\paren{\frac{\partial^{#1} {#2}}{\partial {#3}{}^{#1}}}_{#4}}}
\newcommand{\tpsdiff}[4][{}]{{\paren{\frac{\partial^{#1}}{\partial {#3}{}^{#1}}{#2}}_{#4}}}
\newcommand{\pdiff}[3][{}]{{\frac{\partial^{#1} {#2}}{\partial {#3}{}^{#1}}}}
\newcommand{\diff}[3][{}]{{\frac{\ud^{#1} {#2}}{\ud {#3}{}^{#1}}}}
\newcommand{\psdiff}[3][{}]{{\frac{\partial^{#1}}{\partial {#3}{}^{#1}} {#2}}}
\newcommand{\sdiff}[3][{}]{{\frac{\ud^{#1}}{\ud {#3}{}^{#1}} {#2}}}
\newcommand{\tpddiff}[4][{}]{{\left(\dfrac{\partial^{#1} {#2}}{\partial {#3}{}^{#1}}\right)_{#4}}}
\newcommand{\tpsddiff}[4][{}]{{\paren{\dfrac{\partial^{#1}}{\partial {#3}{}^{#1}}{#2}}_{#4}}}
\newcommand{\pddiff}[3][{}]{{\dfrac{\partial^{#1} {#2}}{\partial {#3}{}^{#1}}}}
\newcommand{\ddiff}[3][{}]{{\dfrac{\ud^{#1} {#2}}{\ud {#3}{}^{#1}}}}
\newcommand{\psddiff}[3][{}]{{\frac{\partial^{#1}}{\partial{}^{#1} {#3}} {#2}}}
\newcommand{\sddiff}[3][{}]{{\frac{\ud^{#1}}{\ud {#3}{}^{#1}} {#2}}}
\usepackage{fancyhdr}
\usepackage{multirow}
\usepackage{fontenc}
%\usepackage{tipa}
\usepackage{ulem}
\usepackage{color}
\usepackage{cancel}
\newcommand{\hcancel}[2][black]{\setbox0=\hbox{#2}%
\rlap{\raisebox{.45\ht0}{\textcolor{#1}{\rule{\wd0}{1pt}}}}#2}
\pagestyle{fancy}
\setlength{\headheight}{67pt}
\fancyhead{}
\fancyfoot{}
\fancyfoot[C]{\thepage}
\fancyhead[R]
{
\psettitlename \\
\psettitlecourse{} Problem Set \psettitlepsetnum \\
\ifx\psettitlersection\empty
\else
Recitation Section \psettitlersection
\fi
}
\renewcommand{\footruleskip}{0pt}
\renewcommand{\headrulewidth}{0.4pt}
\renewcommand{\footrulewidth}{0pt}
\addtolength{\hoffset}{-1.3cm}
\addtolength{\voffset}{-2cm}
\addtolength{\textwidth}{3cm}
\addtolength{\textheight}{2.5cm}
\renewcommand{\footskip}{10pt}
\setlength{\headwidth}{\textwidth}
\setlength{\headsep}{20pt}
\setlength{\marginparwidth}{0pt}
\parindent=0pt
\psetnum{3}
\course{8.311}
\rsection{1}
\name{Yichao Yu}
\renewcommand{\thesection}{\arabic{section}.}
\renewcommand{\thesubsection}{(\alph{subsection})}
\renewcommand{\thesubsubsection}{\roman{subsubsection}.}

\begin{document}
\section{}
\subsection{}
\eqar{
  \vec E_{dipole}=&-\nabla\paren{\frac1{4\pi\varepsilon_0}\frac{\vec r\cdot\vec p}{r^3}}\\
  =&-\frac1{4\pi\varepsilon_0}\nabla\paren{\frac{\vec r\cdot\vec p}{r^3}}\\
  =&-\frac{\vec r\cdot\vec p}{4\pi\varepsilon_0}\nabla\paren{\frac{1}{r^3}}-\frac1{4\pi\varepsilon_0r^3}\nabla\paren{\vec r\cdot\vec p}\\
  =&\frac{\vec r\cdot\vec p}{4\pi\varepsilon_0}\frac{3\hat n}{r^4}-\frac{\vec p}{4\pi\varepsilon_0r^3}\\
  =&\frac{3\hat n\paren{\hat n\cdot\vec p}-\vec p}{4\pi\varepsilon_0r^3}
}
\subsection{}
\eqar{
  &\int\vec J\paren{\vec r'}\paren{\vec r'\cdot\hat n}\ud^3x\\
  =&\frac12\int\vec J\paren{\vec r'}\paren{\vec r'\cdot\hat n}\ud^3x+
  \frac12\int\paren{
    \hat n\times\paren{\vec J\paren{\vec r'}\times\vec r'}-
    \vec r'\paren{\hat n\cdot\vec J\paren{\vec r'}}
  }\ud^3x\\
  =&\hat n\times\frac12\int\vec J\paren{\vec r'}\times\vec r'\ud^3x+
  \frac12\int
    \paren{\vec J\paren{\vec r'}\paren{\vec r'\cdot\hat n}-
    \vec r'\paren{\hat n\cdot\vec J\paren{\vec r'}}}
  \ud^3x
  \intertext{For arbitrary vector $\vec l$}
  &\nabla'\cdot\paren{\paren{\vec l\cdot\vec r'}\paren{\hat n\cdot\vec r'}\vec J\paren{r'}}\\
  =&\paren{\hat n\cdot\vec r'}\vec J\paren{r'}\cdot\nabla'\paren{\vec l\cdot\vec r'}+
  \paren{\vec l\cdot\vec r'}\vec J\paren{r'}\cdot\nabla'\paren{\hat n\cdot\vec r'}+
  \paren{\vec l\cdot\vec r'}\paren{\hat n\cdot\vec r'}\nabla'\cdot\vec J\paren{r'}\\
  =&\paren{\hat n\cdot\vec r'}\vec J\paren{r'}\cdot\vec l+
  \paren{\vec l\cdot\vec r'}\vec J\paren{r'}\cdot\hat n\\
  =&\vec l\cdot\paren{\paren{\hat n\cdot\vec r'}\vec J\paren{r'}+
    \vec r'\vec J\paren{r'}\cdot\hat n}
  \intertext{Integrate both sides}
  0=&\vec l\cdot\int\ud^3x'\paren{\paren{\hat n\cdot\vec r'}\vec J\paren{r'}+
    \vec r'\vec J\paren{r'}\cdot\hat n}\\
  0=&\int\ud^3x'\paren{\paren{\hat n\cdot\vec r'}\vec J\paren{r'}+
    \vec r'\vec J\paren{r'}\cdot\hat n}\\
  &\int\vec J\paren{\vec r'}\paren{\vec r'\cdot\hat n}\ud^3x\\
  =&\vec m\times\vec n
}
\subsection{}
\eqar{
  \frac{4\pi}{\mu_0}\vec B_{dipole}=&\nabla\times\frac{\vec m\times\vec r}{r^3}\\
  =&\vec m\paren{\nabla\cdot\frac{\vec r}{r^3}}-
  \paren{\vec m\cdot\nabla}\frac{\vec r}{r^3}\\
  =&-\vec r\paren{\vec m\cdot\nabla\frac{1}{r^3}}
  -\frac{1}{r^3}\paren{\vec m\cdot\nabla}\vec r\\
  =&\vec r\frac{3\vec m\cdot\hat n}{r^4}-\frac{\vec m}{r^3}\\
  =&\frac{3\hat n\paren{\vec m\cdot\hat n}-\vec m}{r^3}
}
\section{}
\subsection{}
\eqar{
  E=&\int_R^\infty\ud r\int_0^{2\pi}\ud\phi\int_{0}^{\pi}\sin\theta\ud\theta\frac{p^2}{32\pi^2\varepsilon_0r^4}\paren{4\cos^2\theta+\sin^2\theta}\\
  =&\frac{p^2}{16\pi\varepsilon_0}\int_R^\infty\frac{\ud r}{r^4}\int_{0}^{\pi}\ud\theta\sin\theta\paren{4\cos^2\theta+\sin^2\theta}\\
  =&\frac{p^2}{12\pi\varepsilon_0 R^3}
}
\subsection{}
Direvatives of the dipole moment
\eqar{
  \dot\vec p=&\dot p\hat z\\
  \ddot\vec p=&\ddot p\hat z
  \intertext{Magnetic field}
  \vec B=&\frac{\hat z\times\hat n}{4\pi\varepsilon_0}\paren{\frac{\dot p}{r^2}+\frac{\ddot p}{cr}}\\
  =&\frac{\sin\theta\hat\phi}{4\pi\varepsilon_0}\paren{\frac{\dot p}{r^2}+\frac{\ddot p}{cr}}\\
  \intertext{Electric field}
  \vec E=&\frac{3\hat n\paren{\vec p\cdot\hat n}-\vec p}{4\pi\varepsilon_0r^3}
  +\frac{3\hat n\paren{\dot{\vec p}\cdot\hat n}-\dot{\vec p}}{4\pi\varepsilon_0cr^2}
  +\frac{\paren{\ddot{\vec p}\times\hat n}\times\hat n}{4\pi\varepsilon_0c^2r}\\
  =&\frac{3p\cos\theta\hat n-p\hat z}{4\pi\varepsilon_0r^3}
  +\frac{3\dot p\cos\theta\hat n-\dot p\hat z}{4\pi\varepsilon_0cr^2}
  +\frac{\ddot p\sin\theta\hat\phi\times\hat n}{4\pi\varepsilon_0c^2r}\\
  =&\frac{3p\cos\theta\hat n-p\hat z}{4\pi\varepsilon_0r^3}
  +\frac{3\dot p\cos\theta\hat n-\dot p\hat z}{4\pi\varepsilon_0cr^2}
  +\frac{\ddot p\sin\theta\hat\theta}{4\pi\varepsilon_0c^2r}
  \intertext{Since $\hat z=\cos\theta\hat n-\sin\theta\hat\theta$}
  \vec E=&\frac{2p\cos\theta\hat n+p\sin\theta\hat\theta}{4\pi\varepsilon_0r^3}
  +\frac{2\dot p\cos\theta\hat n+\dot p\sin\theta\hat\theta}{4\pi\varepsilon_0cr^2}
  +\frac{\ddot p\sin\theta\hat\theta}{4\pi\varepsilon_0c^2r}\\
  =&\frac{2\cos\theta\hat n}{4\pi\varepsilon_0}\paren{\frac{p}{r^3}+\frac{\dot p}{cr^2}}
  +\frac{\sin\theta\hat\theta}{4\pi\varepsilon_0}\paren{\frac{p}{r^3}+\frac{\dot p}{cr^2}+\frac{\ddot p}{c^2r}}
  \intertext{Energy flux}
  \Phi_E=&\frac{1}{\mu_0 c^2}\int\ud t\int_0^{2\pi}\ud\phi\int_0^{\pi}\ud\theta\sin\theta r^2
  \frac{\sin\theta}{4\pi\varepsilon_0}\paren{\frac{\dot p}{r^2}+\frac{\ddot p}{cr}}\frac{\sin\theta}{4\pi\varepsilon_0}\paren{\frac{p}{r^3}+\frac{\dot p}{cr^2}+\frac{\ddot p}{c^2r}}\\
  =&\frac{1}{6\pi\varepsilon_0}\int\ud t
  \paren{\frac{\dot p}{r}+\frac{\ddot p}{c}}\paren{\frac{p}{r^2}+\frac{\dot p}{cr}+\frac{\ddot p}{c^2}}\\
  =&\frac{1}{6\pi\varepsilon_0}\paren{\frac1{2r}\paren{\frac{p}{r}+\frac{\dot p}{c}}^2}_{t_0}^{t_1}+
  \frac{1}{6\pi\mu_0\varepsilon_0^2}\int\ud t
  \paren{\frac{\dot p}{r}+\frac{\ddot p}{c}}\paren{\frac{\ddot p}{c^2}}\\
  =&\frac{p_2^2-p_1^2}{12\pi\varepsilon_0r^3}+
  \int\frac{\ddot p^2}{6\pi\mu_0\varepsilon_0^2c^3}\ud t
}
The first term corresponds to change in the energy stored in the field.
\section{}
\subsection{}
\eqar{
  r=&\frac{mV_0}{qB_0}
}
\subsection{}
\eqar{
  T=&\frac{\pi m}{qB}
}
\subsection{}
\eqar{
  \diff{W}{t}=&\frac{q^4V_0^2B^2}{6\pi\varepsilon_0m^2c^3}\\
  =&\frac{q^2V_0^4}{6\pi\varepsilon_0c^3R^2}\\
}
\subsection{}
\eqar{
  W=&\frac{q^2V_0^3}{6\varepsilon_0c^3R}\\
  =&\frac{2\pi mV_0^3}{3cR}
}
\subsection{}
\eqar{
  \frac{W}{E_k}=&\frac{2\pi V_0R_{classical}}{3cR}
}
When $R$ is large.
\subsection{}
\section{}
\subsection{}
\eqar{
  p=&Q_0d\sin\omega t\\
  \abs{\diff{W}{t}}=&\frac{Q_0^2d^2\omega^4}{6\pi\varepsilon_0c^3}\abs{\sin^2\omega t}\\
  =&\frac{Q_0^2d^2\omega^4}{12\pi\varepsilon_0c^3}
}
\subsection{}
\eqar{
  E_{rad}=&\frac{Q_0^2d^2\omega^4}{12\pi\varepsilon_0c^3}\frac{2\pi}{\omega}\\
  =&\frac{Q_0^2d^2\omega^3}{6\varepsilon_0c^3}\\
  \frac{4CE_{rad}}{Q_0^2}=&\frac{2d^2C\omega^3}{3\varepsilon_0c^3}\\
  =&\frac{2dAk^3}{3}
}
Therefore if $dk$ and $Ak^2$ are all small (where $k$ is the wave vector) the radiation is small.
\subsection{}
\eqar{
  R_{rad}=&\frac{Q_0^2d^2\omega^4}{12\pi\varepsilon_0c^3}\frac{2}{\omega^2 Q_0^2}\\
  =&\frac{d^2\omega^2}{6\pi\varepsilon_0c^3}
}
\subsection{}
\eqar{
  R_{rad}=&\frac{d^2}{6\pi\varepsilon_0c^3LC}\\
  =&\frac{hd^3}{6\pi\varepsilon_0c^3\varepsilon_0 A_c\mu_0N^2A_L}\\
  =&\mu_0 c\frac{hd^3}{6\pi A_c N^2A_L}
}
\section{}
\subsection{}
\eqar{
  \vec B_{\perp}=&\frac{\mu_0}{4\pi}\paren{\frac1{cr^2}\paren{3\hat n\paren{\dot{\vec m}\cdot\hat n}-\dot{\vec m}}+\frac1{rc^2}\paren{\ddot{\vec m}\times\hat n}\times\hat n}\\
  =&\frac{\mu_0 m_0\omega_0}{4\pi rc}\lparen{\frac1{r}\paren{3\hat n\paren{\paren{\cos\omega_0t\hat y-\sin\omega_0t\hat x}\cdot\hat n}-\paren{\cos\omega_0t\hat y-\sin\omega_0t\hat x}}}\\
  &-\rparen{\frac{\omega_0}{c}\paren{\paren{\cos\omega_0t\hat x+\sin\omega_0t\hat y}\times\hat n}\times\hat n}\\
  =&\frac{\mu_0 m_0\omega_0}{4\pi rc}\lparen{\frac{2\hat e_r}{r}\paren{\cos\omega_0t\sin\theta\sin\phi-\sin\omega_0t\sin\theta\cos\phi}}\\
  &-\frac1{r}\cos\omega_0t\paren{\cos\theta\sin\phi\hat e_\theta+\cos\phi\hat e_\phi}+\frac1{r}\sin\omega_0t\paren{\cos\theta\cos\phi\hat e_\theta-\sin\phi\hat e_\phi}\\
  &-\rparen{\frac{\omega_0}{c}\paren{\cos\omega_0t\paren{-\cos\theta\cos\phi\hat e_\phi-\sin\phi\hat e_\theta}+\sin\omega_0t\paren{-\cos\theta\sin\phi\hat e_\phi+\cos\phi\hat e_\theta}}\times\hat e_r}\\
  =&\frac{\mu_0 m_0\omega_0}{4\pi rc}\lparen{\frac{2\hat e_r\sin\theta\sin\paren{\phi-\omega_0t}-\cos\theta\sin\paren{\phi-\omega_0t}\hat e_\theta
      -\cos\paren{\phi-\omega_0t}\hat e_\phi}{r}}\\
  &+\rparen{\frac{\omega_0}{c}\paren{\cos\theta\cos\paren{\phi-\omega_0t}\hat e_\theta-\sin\paren{\phi-\omega_0t}\hat e_\phi}}
}
\eqar{
  \vec E=&-\frac{\mu_0}{4\pi}\paren{\frac{\ddot{\vec m}}{cr}+\frac{\dot{\vec m}}{r^2}}\times\hat e_r\\
  =&\frac{\mu_0 m_0}{4\pi r}\paren{\frac{\omega_0^2}{c}\paren{\cos\theta\cos\paren{\phi-\omega_0t}\hat e_\theta-\sin\paren{\phi-\omega_0t}\hat e_\phi}
    -\frac{\omega_0}{r}\paren{\cos\theta\sin\paren{\phi-\omega_0t}\hat e_\theta+\cos\paren{\phi-\omega_0t}\hat e_\phi}}\times\hat e_r\\
  =&\frac{\mu_0 m_0\omega_0}{4\pi r}\paren{-\frac{\omega_0}{c}\paren{\cos\theta\cos\paren{\phi-\omega_0t}\hat e_\phi+\sin\paren{\phi-\omega_0t}\hat e_\theta}
    +\frac{1}{r}\paren{\cos\theta\sin\paren{\phi-\omega_0t}\hat e_\phi-\cos\paren{\phi-\omega_0t}\hat e_\theta}}
}
\subsection{}
For radiation part
\eqar{
  \vec E_{rad}=&-\frac{\mu_0 m_0\omega_0^2}{4\pi rc}\paren{\cos\theta\cos\paren{\phi-\omega_0t}\hat e_\phi+\sin\paren{\phi-\omega_0t}\hat e_\theta}
}
Helicity is positive for $\theta>0$ and negative for $\theta<0$\\
Ellipticity is $\abs{\cos\theta}$
\subsection{}
\eqar{
  \frac{\ud W_{rad}}{\ud\Omega\ud t}=&\frac{1}{\mu_0c}\frac{\mu_0^2 m_0^2\omega_0^4}{16\pi^2 r^2c^2}\frac12\paren{\cos^2\theta+1}\\
  =&\frac{\mu_0 m_0^2\omega_0^4\paren{\cos^2\theta+1}}{32\pi^2 r^2c^3}
}
\subsection{}
\eqar{
  \angl{\diff{W}{t}}=&\frac{3\mu_0 m_0^2\omega_0^4}{16\pi r^2c^3}
}
\subsection{}
\eqar{
  \diff{W}{t\ud\Omega}=&-r^3\hat n\times\paren{\paren{\varepsilon_0\vec E\vec E+\frac1{\mu_0}\vec B\vec B}\cdot\hat n}\\
  =&-r^3\paren{\varepsilon_0\hat n\times\vec E\vec E\cdot\hat n+\frac1{\mu_0}\hat n\times\vec B\vec B\cdot\hat n}\\
  =&-\frac{m_0\omega_0r}{2\pi c}\sin\theta\sin\paren{\phi-\omega_0t}\hat n\times\vec B
  \intertext{Ignoring the terms that vanishes for large $r$}
  \diff{W}{t\ud\Omega}=&-\frac{\mu_0 m_0^2\omega_0^3}{8\pi^2c^3}\sin\theta\sin\paren{\phi-\omega_0t}\paren{\cos\theta\cos\paren{\phi-\omega_0t}\hat e_\phi+\sin\paren{\phi-\omega_0t}\hat e_\theta}
  \intertext{Time averaging}
  \angl{\diff{W}{t\ud\Omega}}=&-\frac{\mu_0 m_0^2\omega_0^3}{16\pi^2c^3}\sin\theta\hat e_\theta
}
\subsection{}
After the angular integral, only the $z$ component can be not zero
\eqar{
  \angl{\diff{W}{\Omega}}=&\hat z\frac{\mu_0 m_0^2\omega_0^3}{16\pi^2 c^3}\int\ud\Omega\sin^2\theta\\
  =&\hat z\frac{\mu_0 m_0^2\omega_0^3}{8\pi c^3}
}
\section{}
\subsection{}
\subsection{}
\subsection{}

\end{document}
