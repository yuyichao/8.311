\documentclass[10pt,fleqn]{article}
\newcommand{\name}[1]{\def\psettitlename{#1}}
\newcommand{\course}[1]{\def\psettitlecourse{#1}}
\newcommand{\rsection}[1]{\def\psettitlersection{#1}}
\newcommand{\psetnum}[1]{\def\psettitlepsetnum{#1}}
%\usepackage[journal=rsc]{chemstyle}
%\usepackage{mhchem}
\usepackage{amsmath}
\usepackage{amssymb}
\usepackage{amsfonts}
\usepackage{esint}
\usepackage{bbm}
\usepackage{amscd}
\usepackage{picinpar}
\usepackage[pdftex]{graphicx}
\usepackage{tikz}
\usepackage{indentfirst}
\usepackage{wrapfig}
\usepackage{units}
\usepackage{textcomp}
\usepackage[utf8x]{inputenc}
% \usepackage{feyn}
\usepackage{feynmp}
\usetikzlibrary{
  arrows,
  calc,
  decorations.pathmorphing,
  decorations.pathreplacing,
  decorations.markings,
  fadings,
  positioning,
  shapes
}

\DeclareGraphicsRule{*}{mps}{*}{}
\newcommand{\ud}{\mathrm{d}}
\newcommand{\ue}{\mathrm{e}}
\newcommand{\ui}{\mathrm{i}}
\newcommand{\res}{\mathrm{Res}}
\newcommand{\Tr}{\mathrm{Tr}}
\newcommand{\dsum}{\displaystyle\sum}
\newcommand{\dprod}{\displaystyle\prod}
\newcommand{\dlim}{\displaystyle\lim}
\newcommand{\dint}{\displaystyle\int}
\newcommand{\fsno}[1]{{\!\not\!{#1}}}
\newcommand{\eqar}[1]
{
  \begin{align*}
    #1
  \end{align*}
}
\newcommand{\texp}[2]{\ensuremath{{#1}\times10^{#2}}}
\newcommand{\dexp}[2]{\ensuremath{{#1}\cdot10^{#2}}}
\newcommand{\eval}[2]{{\left.{#1}\right|_{#2}}}
\newcommand{\paren}[1]{{\left({#1}\right)}}
\newcommand{\lparen}[1]{{\left({#1}\right.}}
\newcommand{\rparen}[1]{{\left.{#1}\right)}}
\newcommand{\abs}[1]{{\left|{#1}\right|}}
\newcommand{\sqr}[1]{{\left[{#1}\right]}}
\newcommand{\crly}[1]{{\left\{{#1}\right\}}}
\newcommand{\angl}[1]{{\left\langle{#1}\right\rangle}}
\newcommand{\tpdiff}[4][{}]{{\paren{\frac{\partial^{#1} {#2}}{\partial {#3}{}^{#1}}}_{#4}}}
\newcommand{\tpsdiff}[4][{}]{{\paren{\frac{\partial^{#1}}{\partial {#3}{}^{#1}}{#2}}_{#4}}}
\newcommand{\pdiff}[3][{}]{{\frac{\partial^{#1} {#2}}{\partial {#3}{}^{#1}}}}
\newcommand{\diff}[3][{}]{{\frac{\ud^{#1} {#2}}{\ud {#3}{}^{#1}}}}
\newcommand{\psdiff}[3][{}]{{\frac{\partial^{#1}}{\partial {#3}{}^{#1}} {#2}}}
\newcommand{\sdiff}[3][{}]{{\frac{\ud^{#1}}{\ud {#3}{}^{#1}} {#2}}}
\newcommand{\tpddiff}[4][{}]{{\left(\dfrac{\partial^{#1} {#2}}{\partial {#3}{}^{#1}}\right)_{#4}}}
\newcommand{\tpsddiff}[4][{}]{{\paren{\dfrac{\partial^{#1}}{\partial {#3}{}^{#1}}{#2}}_{#4}}}
\newcommand{\pddiff}[3][{}]{{\dfrac{\partial^{#1} {#2}}{\partial {#3}{}^{#1}}}}
\newcommand{\ddiff}[3][{}]{{\dfrac{\ud^{#1} {#2}}{\ud {#3}{}^{#1}}}}
\newcommand{\psddiff}[3][{}]{{\frac{\partial^{#1}}{\partial{}^{#1} {#3}} {#2}}}
\newcommand{\sddiff}[3][{}]{{\frac{\ud^{#1}}{\ud {#3}{}^{#1}} {#2}}}
\usepackage{fancyhdr}
\usepackage{multirow}
\usepackage{fontenc}
%\usepackage{tipa}
\usepackage{ulem}
\usepackage{color}
\usepackage{cancel}
\newcommand{\hcancel}[2][black]{\setbox0=\hbox{#2}%
\rlap{\raisebox{.45\ht0}{\textcolor{#1}{\rule{\wd0}{1pt}}}}#2}
\pagestyle{fancy}
\setlength{\headheight}{67pt}
\fancyhead{}
\fancyfoot{}
\fancyfoot[C]{\thepage}
\fancyhead[R]
{
\psettitlename \\
\psettitlecourse{} Problem Set \psettitlepsetnum \\
\ifx\psettitlersection\empty
\else
Recitation Section \psettitlersection
\fi
}
\renewcommand{\footruleskip}{0pt}
\renewcommand{\headrulewidth}{0.4pt}
\renewcommand{\footrulewidth}{0pt}
\addtolength{\hoffset}{-1.3cm}
\addtolength{\voffset}{-2cm}
\addtolength{\textwidth}{3cm}
\addtolength{\textheight}{2.5cm}
\renewcommand{\footskip}{10pt}
\setlength{\headwidth}{\textwidth}
\setlength{\headsep}{20pt}
\setlength{\marginparwidth}{0pt}
\parindent=0pt
\psetnum{7}
\input{../head}
\begin{document}
\section{}
\subsection{}
Expand the field in derivatives in $x$ and $y$
\eqar{
  \vec E=&\paren{\vec E_0+\vec E_1}\ue^{\ui kz-\ui\omega t}+\cdots\\
  \vec B=&\paren{\vec B_0+\vec B_1}\ue^{\ui kz-\ui\omega t}+\cdots\\
  \intertext{where $\vec E_0=E_0\paren{\hat x\pm\ui\hat y}$, Since}
  \nabla\cdot\vec E=&0\\
  0=&\nabla\cdot\vec E_0\ue^{\ui kz-\ui\omega t}+\nabla\ue^{\ui kz-\ui\omega t}\cdot\paren{\vec E_0+\vec E_1}\\
  =&\paren{\pdiff{E_0}{x}\pm\ui\pdiff{E_0}{y}}\ue^{\ui kz-\ui\omega t}+\ui k\ue^{\ui kz-\ui\omega t}E_{z_1}\\
  E_{z_1}=&\frac{\ui}{k}\paren{\pdiff{E_0}{x}\pm\ui\pdiff{E_0}{y}}\\
  \vec E=&\paren{E_0\paren{\hat x\pm\ui\hat y}+\frac{\ui}{k}\paren{\pdiff{E_0}{x}\pm\ui\pdiff{E_0}{y}}\hat z}\ue^{\ui kz-\ui\omega t}
  \intertext{$B$ field}
  \vec B=&-\frac{\ui}{\omega}\nabla\times\vec E\\
  =&-\frac{\ui}{\omega}\paren{\ui k\hat z\times\paren{\vec E_0+\vec E_1}\ue^{\ui kz-\ui\omega t}+\paren{\nabla E_0}\times\paren{\hat x\pm\ui\hat y}\ue^{\ui kz-\ui\omega t}}\\
  =&\frac{k}{\omega}\paren{E_0\paren{\hat y\mp\ui\hat x}\pm\frac1k\paren{\pdiff{E_0}{x}\pm\ui\pdiff{E_0}{y}}\hat z}\ue^{\ui kz-\ui\omega t}\\
  =&\mp\ui\frac{k}{\omega}\paren{E_0\paren{\hat x\pm\ui\hat y}+\frac{\ui}k\paren{\pdiff{E_0}{x}\pm\ui\pdiff{E_0}{y}}\hat z}\ue^{\ui kz-\ui\omega t}\\
  =&\mp\ui\sqrt{\mu\varepsilon}\vec E
}
\subsection{}
Angular momentum density
\eqar{
  \vec l=&\varepsilon_0\vec r\times\paren{\vec E\times\vec B}\\
  \angl{l_z}=&\frac{\hat z}2\cdot\varepsilon_0\vec r\times\Re\paren{\vec E\times\vec B^*}\\
  =&\frac{\hat z}2\cdot\varepsilon_0\vec r\times\Re\paren{\vec E_0\times\vec B_1^*+\vec E_1\times\vec B_0^*}\\
  =&\mp\frac{\hat z}2\cdot\sqrt{\mu_0\varepsilon_0}\varepsilon_0\vec r\times\Re\paren{\ui\paren{\vec E_0\times\vec E_1^*-\vec E_0^*\times\vec E_1}}\\
  =&\pm\hat z\cdot\sqrt{\mu_0\varepsilon_0}\varepsilon_0\vec r\times\Im\paren{\vec E_0\times\vec E_1^*}\\
  =&\pm\hat z\cdot\sqrt{\mu_0\varepsilon_0}\varepsilon_0\vec r\times\Im\paren{E_0\paren{\hat x\pm\ui\hat y}\times\frac{-\ui}{k}\paren{\pdiff{E_0}{x}\mp\ui\pdiff{E_0}{y}}\hat z}\\
  =&\mp\frac{\varepsilon_0E_0}{\omega}\Re\paren{\paren{-x\mp\ui y}\paren{\pdiff{E_0}{x}\mp\ui\pdiff{E_0}{y}}}\\
  =&\pm\frac{\varepsilon_0E_0}{\omega}\paren{x\pdiff{E_0}{x}+y\pdiff{E_0}{y}}\\
  =&\pm\frac{\varepsilon_0}{2\omega}\paren{x\pdiff{}{x}+y\pdiff{}{y}}E_0^2
  \intertext{$1$D density}
  \angl{L_z}=&\pm\int\ud\sigma\frac{1}{2\omega}\paren{x\pdiff{}{x}+y\pdiff{}{y}}u\\
  =&\pm\int\ud\sigma\frac{u}{\omega}\\
  =&\pm\frac{U}{\omega}
}
This means a circularly polarized photon of energy $\hbar\omega$ has angular momentum along it's propagation direction of $\pm\hbar$. The transverse component of angular momentum vanish because of symmetry.

\section{}
\subsection{}
\subsection{}

\end{document}
