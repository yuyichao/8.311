\documentclass[10pt,fleqn]{article}
\newcommand{\name}[1]{\def\psettitlename{#1}}
\newcommand{\course}[1]{\def\psettitlecourse{#1}}
\newcommand{\rsection}[1]{\def\psettitlersection{#1}}
\newcommand{\psetnum}[1]{\def\psettitlepsetnum{#1}}
% \usepackage[journal=rsc]{chemstyle}
% \usepackage{mhchem}
\usepackage{amsmath}
\usepackage{amssymb}
\usepackage{amsfonts}
\usepackage{esint}
\usepackage{bbm}
\usepackage{amscd}
\usepackage{picinpar}
\usepackage[pdftex]{graphicx}
\usepackage{tikz}
\usepackage{indentfirst}
\usepackage{wrapfig}
\usepackage{units}
\usepackage{textcomp}
\usepackage[utf8x]{inputenc}
% \usepackage{feyn}
\usepackage{feynmp}
\usetikzlibrary{
  arrows,
  calc,
  decorations.pathmorphing,
  decorations.pathreplacing,
  decorations.markings,
  fadings,
  positioning,
  shapes
}

\DeclareGraphicsRule{*}{mps}{*}{}
\newcommand{\ud}{\mathrm{d}}
\newcommand{\ue}{\mathrm{e}}
\newcommand{\ui}{\mathrm{i}}
\newcommand{\res}{\mathrm{Res}}
\newcommand{\Tr}{\mathrm{Tr}}
\newcommand{\dsum}{\displaystyle\sum}
\newcommand{\dprod}{\displaystyle\prod}
\newcommand{\dlim}{\displaystyle\lim}
\newcommand{\dint}{\displaystyle\int}
\newcommand{\fsno}[1]{{\!\not\!{#1}}}
\newcommand{\eqar}[1]
{
  \begin{align*}
    #1
  \end{align*}
}
\newcommand{\texp}[2]{\ensuremath{{#1}\times10^{#2}}}
\newcommand{\dexp}[2]{\ensuremath{{#1}\cdot10^{#2}}}
\newcommand{\eval}[2]{{\left.{#1}\right|_{#2}}}
\newcommand{\paren}[1]{{\left({#1}\right)}}
\newcommand{\lparen}[1]{{\left({#1}\right.}}
\newcommand{\rparen}[1]{{\left.{#1}\right)}}
\newcommand{\abs}[1]{{\left|{#1}\right|}}
\newcommand{\sqr}[1]{{\left[{#1}\right]}}
\newcommand{\crly}[1]{{\left\{{#1}\right\}}}
\newcommand{\angl}[1]{{\left\langle{#1}\right\rangle}}
\newcommand{\tpdiff}[4][{}]{{\paren{\frac{\partial^{#1} {#2}}{\partial {#3}{}^{#1}}}_{#4}}}
\newcommand{\tpsdiff}[4][{}]{{\paren{\frac{\partial^{#1}}{\partial {#3}{}^{#1}}{#2}}_{#4}}}
\newcommand{\pdiff}[3][{}]{{\frac{\partial^{#1} {#2}}{\partial {#3}{}^{#1}}}}
\newcommand{\diff}[3][{}]{{\frac{\ud^{#1} {#2}}{\ud {#3}{}^{#1}}}}
\newcommand{\psdiff}[3][{}]{{\frac{\partial^{#1}}{\partial {#3}{}^{#1}} {#2}}}
\newcommand{\sdiff}[3][{}]{{\frac{\ud^{#1}}{\ud {#3}{}^{#1}} {#2}}}
\newcommand{\tpddiff}[4][{}]{{\left(\dfrac{\partial^{#1} {#2}}{\partial {#3}{}^{#1}}\right)_{#4}}}
\newcommand{\tpsddiff}[4][{}]{{\paren{\dfrac{\partial^{#1}}{\partial {#3}{}^{#1}}{#2}}_{#4}}}
\newcommand{\pddiff}[3][{}]{{\dfrac{\partial^{#1} {#2}}{\partial {#3}{}^{#1}}}}
\newcommand{\ddiff}[3][{}]{{\dfrac{\ud^{#1} {#2}}{\ud {#3}{}^{#1}}}}
\newcommand{\psddiff}[3][{}]{{\frac{\partial^{#1}}{\partial{}^{#1} {#3}} {#2}}}
\newcommand{\sddiff}[3][{}]{{\frac{\ud^{#1}}{\ud {#3}{}^{#1}} {#2}}}
\usepackage{fancyhdr}
\usepackage{multirow}
\usepackage{fontenc}
% \usepackage{tipa}
\usepackage{ulem}
\usepackage{color}
\usepackage{cancel}
\newcommand{\hcancel}[2][black]{\setbox0=\hbox{#2}%
  \rlap{\raisebox{.45\ht0}{\textcolor{#1}{\rule{\wd0}{1pt}}}}#2}
\pagestyle{fancy}
\setlength{\headheight}{67pt}
\fancyhead{}
\fancyfoot{}
\fancyfoot[C]{\thepage}
\fancyhead[R]
{
  \psettitlename \\
  \psettitlecourse{} Problem Set \psettitlepsetnum \\
  \ifx\psettitlersection\empty
  \else
  Recitation Section \psettitlersection
  \fi
}
\renewcommand{\footruleskip}{0pt}
\renewcommand{\headrulewidth}{0.4pt}
\renewcommand{\footrulewidth}{0pt}
\addtolength{\hoffset}{-1.3cm}
\addtolength{\voffset}{-2cm}
\addtolength{\textwidth}{3cm}
\addtolength{\textheight}{2.5cm}
\renewcommand{\footskip}{10pt}
\setlength{\headwidth}{\textwidth}
\setlength{\headsep}{20pt}
\setlength{\marginparwidth}{0pt}
\parindent=0pt
\psetnum{11}
\course{8.311}
\rsection{1}
\name{Yichao Yu}
\renewcommand{\thesection}{\arabic{section}.}
\renewcommand{\thesubsection}{(\alph{subsection})}
\renewcommand{\thesubsubsection}{\roman{subsubsection}.}

\begin{document}
\section{}
\subsection{}
\eqar{
  \mu_0\pdiff{}{t}\vec J=&\nabla^2\vec E_{\perp}-\frac{1}{c^2}\pdiff[2]{\vec E_{\perp}}{t}
  \intertext{On the shell}
  \mu_0\pdiff{}{t}\tilde J_\phi=&\pdiff[2]{}{r}\tilde E_{\phi}\\
  \mu_0\pdiff{}{t}\tilde J_\phi=&-\ui\omega\mu_0\sigma\Omega_0 R\delta\paren{r-R}\sin\theta\ue^{-\ui\omega t}\\
  \pdiff[2]{}{r}\tilde E_{\phi}=&kA\sin\theta\ue^{-\ui\omega t}\delta\paren{r-R}\\
  &\paren{-\frac{\ue^{\ui x}}{x^5}\paren{\ui x^2-2x-2\ui}\paren{\sin x-x\cos x}
    +\frac{\ue^{\ui x}}{x^5}\paren{x+\ui}\paren{2x\cos x-2\sin x+x^2\sin x}}
  \intertext{where $x=kr$}
  =&Ak\sin\theta\ue^{-\ui\omega t}\delta\paren{r-R}\frac{\ui}{x^2}\\
  A=&-c\mu_0\sigma\Omega_0 R\paren{kR}^2
}
\subsection{}
With $X=kR$
\eqar{
  \tilde E_\phi\paren{R}=&c\mu_0\sigma\Omega_0 RX^2\ue^{\ui X}\frac{\sin X-X\cos X}{X^2}\frac{X+\ui}{X^2}\sin\theta\ue^{-\ui\omega t}\\
  =&c\mu_0\sigma\Omega_0 R\ue^{\ui X}\frac{\sin X-X\cos X}{X^2}\paren{X+\ui}\sin\theta\ue^{-\ui\omega t}
  \intertext{Average power}
  P=&-R^2\int\ud\Omega\Re\paren{c\mu_0\sigma^2\Omega_0^2 R^2\ue^{\ui X}\frac{\sin X-X\cos X}{X^2}\paren{X+\ui}\sin^2\theta}\\
  =&\frac{8\pi}{3} c\mu_0\sigma^2\Omega_0^2 R^4\frac{\paren{\sin X-X\cos X}^2}{X^2}\\
  =&\frac{8\pi}{3} c\mu_0\sigma^2\Omega_0^2 R^2\frac{\paren{\sin kR-kR\cos kR}^2}{k^2}
  \intertext{Magnetostatic energy}
  U_B=&\mu_0\frac{4\pi R^3}{9}\sigma^2\Omega_0^2R^2
  \intertext{Normalized radiation}
  p=&6\pi\frac{\paren{\sin kR-kR\cos kR}^2}{\paren{kR}^3}
}
\begin{tikzpicture}
  \draw (0, 0) -- plot[domain=0.01:10,samples=1000]
  function {6 * (sin(x) - x * cos(x))**2 / (x**3)};
  \draw[->, line width=1] (0, 0) -- (11, 0);
  \draw[->, line width=1] (0, 0) -- (0, 4);
\end{tikzpicture}
\subsection{}
For small $kR$, $p\propto \paren{kR}^3$. This makes sense since it is the same scaling with a dipole radiation with constant change rate (i.e. one less factor of $\omega$)
\subsection{}
The pattern for larger $kR$ appears because of the interference of radiation from different part of the sphere. The maximum radiation appears when $kR\approx2.46$.

\end{document}
