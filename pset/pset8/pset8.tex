\documentclass[10pt,fleqn]{article}
\newcommand{\name}[1]{\def\psettitlename{#1}}
\newcommand{\course}[1]{\def\psettitlecourse{#1}}
\newcommand{\rsection}[1]{\def\psettitlersection{#1}}
\newcommand{\psetnum}[1]{\def\psettitlepsetnum{#1}}
%\usepackage[journal=rsc]{chemstyle}
%\usepackage{mhchem}
\usepackage{amsmath}
\usepackage{amssymb}
\usepackage{amsfonts}
\usepackage{esint}
\usepackage{bbm}
\usepackage{amscd}
\usepackage{picinpar}
\usepackage[pdftex]{graphicx}
\usepackage{tikz}
\usepackage{indentfirst}
\usepackage{wrapfig}
\usepackage{units}
\usepackage{textcomp}
\usepackage[utf8x]{inputenc}
% \usepackage{feyn}
\usepackage{feynmp}
\usetikzlibrary{
  arrows,
  calc,
  decorations.pathmorphing,
  decorations.pathreplacing,
  decorations.markings,
  fadings,
  positioning,
  shapes
}

\DeclareGraphicsRule{*}{mps}{*}{}
\newcommand{\ud}{\mathrm{d}}
\newcommand{\ue}{\mathrm{e}}
\newcommand{\ui}{\mathrm{i}}
\newcommand{\res}{\mathrm{Res}}
\newcommand{\Tr}{\mathrm{Tr}}
\newcommand{\dsum}{\displaystyle\sum}
\newcommand{\dprod}{\displaystyle\prod}
\newcommand{\dlim}{\displaystyle\lim}
\newcommand{\dint}{\displaystyle\int}
\newcommand{\fsno}[1]{{\!\not\!{#1}}}
\newcommand{\eqar}[1]
{
  \begin{align*}
    #1
  \end{align*}
}
\newcommand{\texp}[2]{\ensuremath{{#1}\times10^{#2}}}
\newcommand{\dexp}[2]{\ensuremath{{#1}\cdot10^{#2}}}
\newcommand{\eval}[2]{{\left.{#1}\right|_{#2}}}
\newcommand{\paren}[1]{{\left({#1}\right)}}
\newcommand{\lparen}[1]{{\left({#1}\right.}}
\newcommand{\rparen}[1]{{\left.{#1}\right)}}
\newcommand{\abs}[1]{{\left|{#1}\right|}}
\newcommand{\sqr}[1]{{\left[{#1}\right]}}
\newcommand{\crly}[1]{{\left\{{#1}\right\}}}
\newcommand{\angl}[1]{{\left\langle{#1}\right\rangle}}
\newcommand{\tpdiff}[4][{}]{{\paren{\frac{\partial^{#1} {#2}}{\partial {#3}{}^{#1}}}_{#4}}}
\newcommand{\tpsdiff}[4][{}]{{\paren{\frac{\partial^{#1}}{\partial {#3}{}^{#1}}{#2}}_{#4}}}
\newcommand{\pdiff}[3][{}]{{\frac{\partial^{#1} {#2}}{\partial {#3}{}^{#1}}}}
\newcommand{\diff}[3][{}]{{\frac{\ud^{#1} {#2}}{\ud {#3}{}^{#1}}}}
\newcommand{\psdiff}[3][{}]{{\frac{\partial^{#1}}{\partial {#3}{}^{#1}} {#2}}}
\newcommand{\sdiff}[3][{}]{{\frac{\ud^{#1}}{\ud {#3}{}^{#1}} {#2}}}
\newcommand{\tpddiff}[4][{}]{{\left(\dfrac{\partial^{#1} {#2}}{\partial {#3}{}^{#1}}\right)_{#4}}}
\newcommand{\tpsddiff}[4][{}]{{\paren{\dfrac{\partial^{#1}}{\partial {#3}{}^{#1}}{#2}}_{#4}}}
\newcommand{\pddiff}[3][{}]{{\dfrac{\partial^{#1} {#2}}{\partial {#3}{}^{#1}}}}
\newcommand{\ddiff}[3][{}]{{\dfrac{\ud^{#1} {#2}}{\ud {#3}{}^{#1}}}}
\newcommand{\psddiff}[3][{}]{{\frac{\partial^{#1}}{\partial{}^{#1} {#3}} {#2}}}
\newcommand{\sddiff}[3][{}]{{\frac{\ud^{#1}}{\ud {#3}{}^{#1}} {#2}}}
\usepackage{fancyhdr}
\usepackage{multirow}
\usepackage{fontenc}
%\usepackage{tipa}
\usepackage{ulem}
\usepackage{color}
\usepackage{cancel}
\newcommand{\hcancel}[2][black]{\setbox0=\hbox{#2}%
\rlap{\raisebox{.45\ht0}{\textcolor{#1}{\rule{\wd0}{1pt}}}}#2}
\pagestyle{fancy}
\setlength{\headheight}{67pt}
\fancyhead{}
\fancyfoot{}
\fancyfoot[C]{\thepage}
\fancyhead[R]
{
\psettitlename \\
\psettitlecourse{} Problem Set \psettitlepsetnum \\
\ifx\psettitlersection\empty
\else
Recitation Section \psettitlersection
\fi
}
\renewcommand{\footruleskip}{0pt}
\renewcommand{\headrulewidth}{0.4pt}
\renewcommand{\footrulewidth}{0pt}
\addtolength{\hoffset}{-1.3cm}
\addtolength{\voffset}{-2cm}
\addtolength{\textwidth}{3cm}
\addtolength{\textheight}{2.5cm}
\renewcommand{\footskip}{10pt}
\setlength{\headwidth}{\textwidth}
\setlength{\headsep}{20pt}
\setlength{\marginparwidth}{0pt}
\parindent=0pt
\psetnum{8}
\course{8.311}
\rsection{1}
\name{Yichao Yu}
\renewcommand{\thesection}{\arabic{section}.}
\renewcommand{\thesubsection}{(\alph{subsection})}
\renewcommand{\thesubsubsection}{\roman{subsubsection}.}

\begin{document}
\section{}
\eqar{
  S_{1,0}=&\frac12\frac{\omega ka^2\mu}{\pi^2}H_0^2\sin^2\frac{\pi x}{a}\\
  P_{1,0}=&\frac14\frac{\omega ka^3b\mu}{\pi^2}H_0^2\\
  -\diff{P}{z}=&\frac{1}{2\sigma\delta}\int\abs{\vec n\times\vec H}^2\ud l\\
  =&\frac{H_0^2}{4\sigma\delta}\paren{2b+a+\frac{k^2a^3}{\pi^2}}\\
  \beta_{1,0}=&\frac{\pi^2}{2\mu_0ka^3b\sqrt{2\mu_c\sigma\omega}}\paren{2b+a+\frac{k^2a^3}{\pi^2}}\\
  =&\frac{\pi^2}{2\mu_0a^3b\sqrt{2\mu_c\mu\varepsilon\sigma\omega\paren{\omega^2-\omega_{1,0}^2}}}\paren{2b+a+\frac{\mu\varepsilon\paren{\omega^2-\omega_{1,0}^2}a^3}{\pi^2}}
}

\section{}
\eqar{
  B_z=&B_0\cos\frac{\pi x}{a}\sin\frac{\pi z}{d}\ue^{-\ui\omega t}\\
  B_x=&-\frac{a}{d}B_0\sin\frac{\pi x}{a}\cos\frac{\pi z}{d}\ue^{-\ui\omega t}\\
  E_y=&\ui\frac{\omega a}{\pi}B_0\sin\frac{\pi x}{a}\sin\frac{\pi z}{d}\ue^{-\ui\omega t}\\
  U=&\frac{a^2d}{8\mu}\paren{B_0^2+B_0^2\frac{a^2}{d^2}+\mu\varepsilon\frac{\omega^2a^2}{\pi^2}B_0^2}\\
  =&\frac{a^2dB_0^2}{4\mu}\paren{1+\frac{a^2}{d^2}}\\
  P=&\frac{B_0^2a^2}{2\mu^2\sigma\delta}\paren{\frac{d}{a}
    +\frac{a^2}{d^2}+\frac{a}{2d}}\\
  Q=&\omega\frac{U}{P}\\
  =&\frac{\mu\sigma\delta\omega d}{2}
  \paren{1+\frac{a^2}{d^2}}\paren{\frac{d}{a}+\frac{a^2}{d^2}+\frac{a}{2d}}^{-1}
}

\section{}
For each solution to the triangular waveguide, mirroring the solution along the diagonal should always create a valid solution for the square waveguide. Therefore, we should be able to construct all solutions to the triangular waveguide from the solutions of the square ones.\\
For TM modes,
\eqar{
  E^{mn}_z=&E_0\sin\frac{m\pi x}{a}\sin\frac{n\pi x}{a}
  \intertext{Since $E_z=0$ on the diagonal}
  E_z^{mn,tri}=&E_z^{mn}-E_z^{nm}
  \intertext{Since $E_z^{mn}$ is symmetric for $m$, $n$, we should have $m > n$ for non-vanishing solution (and to avoid double counting). The cutoff frequencies are the same with the ones for square waveguide}
  \omega_{mn}=&\frac{c\pi}{a}\sqrt{m^2+n^2}
}
For TE modes,
\eqar{
  B^{mn}_z=&E_0\cos\frac{m\pi x}{a}\cos\frac{n\pi x}{a}
  \intertext{Since $\partial_nB_z=0$ on the diagonal}
  B_z^{mn,tri}=&B_z^{mn}+B_z^{nm}
  \intertext{Since $B_z^{mn}$ is symmetric for $m$, $n$, we should have $m \geqslant n$ to avoid double counting. The cutoff frequencies are the same with the ones for square waveguide}
  \omega_{mn}=&\frac{c\pi}{a}\sqrt{m^2+n^2}
}

\end{document}
